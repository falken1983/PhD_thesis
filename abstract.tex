%sample file for Modelica 2011 Abstract page
\documentclass[11pt,a4paper]{article}
\usepackage{graphicx}
% uncomment according to your operating system:
% ------------------------------------------------
\usepackage[latin1]{inputenc}    %% european characters can be used (Windows, old Linux)
%\usepackage[utf8]{inputenc}     %% european characters can be used (Linux)
%\usepackage[applemac]{inputenc} %% european characters can be used (Mac OS)
% ------------------------------------------------
\usepackage[T1]{fontenc}   %% get hyphenation and accented letters right
\usepackage{mathptmx}      %% use fitting times fonts also in formulas
% do not change these lines:
\pagestyle{empty}                %% no page numbers!
\usepackage[left=35mm, right=35mm, top=15mm, bottom=20mm, noheadfoot]{geometry}
%% please don't change geometry settings!


% begin the document
\begin{document}
\thispagestyle{empty}

\title{\textbf{Efficient Methods for Calibrating and \\Pricing Interest
    Rate Options}}
\author{Llu\'\i s Navarro Girb\'es}
\date{} % <--- leave date empty
\maketitle\thispagestyle{empty} %% <-- you need this for the first page
\begin{center}
\textbf{Abstract}
\end{center}
This PhD thesis deals with the valuation and hedging of interest rate
derivatives, a major topic in Financial Economics. Chapters 1, 2 and 3
contain a self contained summary of previous literature, while the
rest of the thesis present the new contributions of this research
work. 

In particular, in Chapter 4 we review the general features of the
geometric view of the HJM models, introducing the concept of
consistent families whith this class of models. 
 
In Chapters 5 and 6 we use the ideas of Chapter 4 to propose
multi-objective extensions of several calibration methods. As a
consequence, a consistent framework for the calibration of vanilla
interest rate derivatives is developed.

Finally, new pricing methods are introduced in the Chapter 7 as an
application of the findings of previous chapters. In particular,
several discretization and simulation techniques to the valuation of
vanilla caps, bond options and binary caps are derived.
\end{document}
