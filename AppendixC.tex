\chapter{Pricing Options with a Consistent HJM Model}
\section*{Digital Caps}
Consider a general HJM model under the risk-neutral measure
$\mathbb{Q}$ specified by (\ref{eqHJM:1}). We also assume that 
$$
\sigma(t,T)= \left[\:\sigma_1(t,T)\:\dots\:\sigma_q(t,T)\:\right]
$$
are deterministic functions of $(t,T)$, and hence forward rates
$F(t,T)$ are Gaussian distributed. Let us consider now a
European-style binary option, with expiration date $T$ and exercise
price $K$, on an underlying bond with maturity $S$ (where of course
$T<S$). The following general pricing formula may be derived:
\begin{propos} The price, at $t=0$ of the security
$$
h(T)=P(T,S) \mathbbm{1}_{ \{ P(T,S)-K \geq 0\} }
$$
is given by
\begin{equation} 
% \label{eqGaussianFROptionsFormula:1}
V(h,0)=P(0,S)N(d_+)
\end{equation}
where
\begin{equation}
% \label{d+-}
\begin{array}{rcl}
d_+ & := & \displaystyle \frac{\log\left(\frac{P(0,S)}{K
      P(0,T)}\right) + \frac{1}{2}\vartheta^2(T,S)}{\vartheta(T,S)}\; , \\ 
\vartheta^2(T,S) & := &\displaystyle \int_0^T \| \varsigma(u;T,S)
\|^2\:du\; ;
\end{array}
\end{equation}
and,
\begin{equation}
% \label{GaussianForwardVol2}
\varsigma(t;T,S) := S(t,S)-S(t,T)=-\int_T^S \sigma(t,s)\: ds.
\end{equation}
\end{propos}
\begin{demo}
Let us start with the fundamental arbitrage-free equation
\begin{equation}
% \label{Decomp}
\label{DigitalBond}
V=\mathbb{E}\left[ D(0,T) P(T,S)\mathbbm{1}_{\{P(T,S)\geq
    K\}}\right] 
\end{equation}
where we are taking the expectations with respect the equivalent
martingale measure $\mathbb{Q}$ associated to the money-market
numeraire $B(\cdot)$.

The Radon-Nikodym derivative that changes $S$-forward
measure $\mathbb{Q}^S$ into the money-market measure $\mathbb{Q}$ will be
given by
$$
\lambda^S(T)=\frac{d\mathbb{Q}^S}{d\mathbb{Q}}=\frac{P(T,S)/P(0,S)}{B(T)/B(0)}=\frac{D(0,T)P(T,S)}{P(0,S)}.
$$
Substituting into (\ref{DigitalBond}), and combining with the
measurability at $t=0$ of $P(0,S)$ 
\begin{eqnarray} 
\nonumber
V &=& \mathbb{E}\left[ P(0,S) \lambda^S(T) \mathbbm{1}_{\{P(T,S)\geq
    K\}}\right] = \\
\nonumber
 &=& P(0,S) \mathbb{Q}^S\left( P(T,S)\geq K\right).
\end{eqnarray}
Now we have the value $V$ for the call option in terms of the forward
measure $\mathbb{Q}^S$. % Let us start with the
% probability computations referred to the $T$-forward measure. Note
% that the probability may be written as
% $$
% \mathbb{Q}^T(P(T,S)\geq K)=\mathbb{Q}^T\left(\frac{P(T,S)}{P(T,T)}\geq K\right)
% =\mathbb{Q}^T\left(\log\frac{P(T,S)}{P(T,T)}\geq \log K\right)$$
% Consider the ``discounted'' process
% $$
% X_{S,T}(t):=\frac{P(t,S)}{P(t,T)},
% $$
% with terminal value $X_{S,T}(T)=P(T,S)/P(T,T)$. By taking differentials
% under the risk-neutral measure $\mathbb{Q}$ we have
% \begin{equation}
% \nonumber
% \begin{array}{rcl}
% \displaystyle d\left( \frac{P(t,S)}{P(t,T)}\right) &=&\displaystyle \frac{1}{P(t,T)}dP(t,S)-\frac{P(t,S)}{(P(t,T))^2}dP(t,T)+dP(t,S)\cdot d\left(\frac{1}{P(t,T)}\right)=\\ 
% &=& \{ \dots \} dt + X\left\{ \left(S(t,S)-S(t,T)\right)dW(t)\right\}\\
% dX&=& \{ \dots \} dt+X\varsigma(t;T,S)dW(t).
% \end{array}
% \end{equation}
% For the second stage we have used equation (\ref{BondRiskNeutral})
% applied to the discount bonds $P(t,S)$ and $P(t,T)$. Recall that
% $\mathbb{Q}^T$ is a martingale measure and the multidimensional
% Girsanov's Theorem\footnote{See [Ref.] for a detailed discussion of it.}
% which locally induces the change into this $T$-forward measure, does
% not affect the difussion coefficient of the initially taken
% differential. Therefore we have 
% $$
% dX_{T,S}(t)=X_{T,S}(t)\varsigma(t;T,S)dW^T(t).
% $$
% Let us introduce the auxiliary process:
% $$
% Y_{T,S}(t)=\log X_{T,S}(t)
% $$
% By means of the multidimensional It\^o Lemma, it is not difficult to
% prove that the random variable $Y_{T,S}(T)$ distributes like
% $$
% Y_{T,S}(T) \sim\mathcal{N}\left( \log \frac{P(0,S)}{P(0,T)}-\frac{1}{2}\vartheta^2(T,S), \vartheta^2(T,S)\right),
% $$
% where $\vartheta^2(T,S)=\int_0^T \| \varsigma(u;T,S) \|^2 \:du$. Now
% the computation of the probability under the $T$-forward measure is
% straightforward:
% $$
% \mathbb{Q}^T(P(T,S)\geq K)=\mathbb{Q}^T(Y_{T,S}(T)\geq \log K)=N(d_-)
% $$
For computing the probability $\mathbb{Q}^S$, first, note the following:
$$
\mathbb{Q}^S(P(T,S)\geq K)=\mathbb{Q}^S\left(\frac{P(T,T)}{P(T,S)}\leq
  \frac{1}{K}\right) =\mathbb{Q}^S\left(\log\frac{P(T,T)}{P(T,S)}\leq
  -\log K\right).$$
We remark that it is enough to introduce the auxiliary processes,
$$
W_{T,S}(t):=\frac{P(t,T)}{P(t,S)},
$$
and,
$$
Z_{T,S}(t):=\log W_{T,S}(t),
$$
to conclude that $Z_{T,S}(T)$ distributes like
$$
Z_{T,S}(T) \sim\mathcal{N}\left( \log \frac{P(0,T)}{P(0,S)}-\frac{1}{2}\vartheta^2(T,S), \vartheta^2(T,S)\right),
$$
and then
$$
\mathbb{Q}^S(Z_{T,S}(T)\leq -\log K)=N(d_+).
$$
\end{demo}
\begin{lema}
The price at $t=0$ of the security 
$$
h(T)=P(T,S) \mathbbm{1}_{ \{ P(T,S)-K < 0 \} }
$$
is given by
\begin{equation}
% \label{GaussianForwardPut}
\Pi(h,0)=P(0,S) N(-d_+)
\end{equation}
where the quantities $d_+$ are completely determined by the
identities (\ref{d+-}) to (\ref{GaussianForwardVol2}).  
\end{lema}
\begin{demo}
First consider the identity option at time $t=0$
\begin{equation}
\nonumber
\begin{split}
V+\Pi & =\mathbb{E} \left[ D(0,T) P(T,S)\left( \mathbbm{1}_{ \{ P(T,S)-K\geq
      0\} }+\mathbbm{1}_{ \{ P(T,S)-K<0 \} } \right) \right]\\ 
& =\mathbb{E}\left[ D(0,T)P(T,S) \right],
\end{split}
\end{equation}
under the risk-neutral martingale measure. 
% By equation (\ref{ArbitrageFree}) we have $$\mathbb{E}\left[ D(0,T)K
% \right]=K P(0,T).$$ However, we have the problem of the correlation
% between the discounting factor and the payoff factor for the first
% term $$\mathbb{E}\left[ D(0,T)P(T,S) \right].$$ We can circumvent this
% problem
By using the \emph{Change of Numeraire Theorem} in an identical way to
that shown by Proposition 7 in this appendix, recall first that the
likelihood 
$$
\lambda^S(T)=\frac{d\mathbb{Q}^S}{d\mathbb{Q}}=\frac{P(T,S)/P(0,S)}{B(T)/B(0)}=\frac{D(0,T)P(T,S)}{P(0,S)},
$$
induces the change of the $S$-forward measure into the risk-neutral
measure. Therefore we have:
$$
\mathbb{E}\left[ D(0,T)P(T,S) \right]=\mathbb{E}\left[
  \lambda^S(T)P(0,S) \right]=\mathbb{E}^S\left[
  P(0,S) \right],
$$
and then the relation:
\begin{equation}
% \label{PutCall}
V+\Pi=P(0,S)
\end{equation}
is inferred. Finally,
\begin{equation}
\label{DigPut}
\Pi = P(0,S) - V = P(0,S) \left( 1 - N(d_+) \right) = P(0,S) N(-d_+)
\end{equation}
\end{demo}
 \begin{corol}[Digital Caplet Pricing for Gaussian Forward
  Rates.] The price at $t=0$ of the binary $j$-caplet, $\delta_j$, with
  payoff:
$$
h_{\delta_j}(x_j)=\mathbbm{1}_{ \{ L_j(x_{j-1})-K \} } 
$$
is given by
\begin{equation}
\label{GaussianForwardBinCaplet}
\delta_j(h_{\delta_j},0)=(1+\tau_j K) P_j(0) N(-d_+)
\end{equation} 
where
\begin{eqnarray}
\label{capletd+} d_{+} & := & \displaystyle \frac{\log\left(\frac{P_j(0)}{\kappa 
      P_{j-1}(0)}\right)+\frac{1}{2}\vartheta^2(0,x_{j-1})}{\vartheta(0,x_{j-1})}\;  , \\  
\label{vol1} \vartheta^2(0,x_{j-1}) & := &\displaystyle
\int_0^{x_{j-1}} \| \varsigma(u;x_{j-1},x_j) \|^2\:du\; ;\\   
\label{vol2} \varsigma(t;x_{j-1},x_j) &  := & -\displaystyle
\int_{x_{j-1}}^{x_j} \sigma(t,s)\: ds.   
\end{eqnarray}
% \end{equation}
\end{corol}
% and $\kappa=(1+\tau_j K)^{-1}$.
% % \begin{propos}[Digital Caplets under Gaussian Forward Rates.] The
% %   price at $t=

% \end{propos}
% % \end{document}% \end{document}% \end{document}% \end{document}% \end{document}
