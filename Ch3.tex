\chapter{Interest Rate Caps}
Interest rate caps are widely traded OTC interest rate derivatives. An
interest rate cap is a financial insurance which protects you from
having to pay more than a predetermined rate,   
therefore, a cap is insurance against rising interest rates.
\section{The Market Practice for Plain Vanilla Caps}
In this section we discuss vanilla interest rate caps and the market
practice for quoting these instruments. For concreteness suppose the
underlying rate is the simply-compounded forward LIBOR rate
$L(t;\cdot,\cdot)$ introduced in Sect. 1.1. Let suppose that we are
standing at time $t=0$. We consider a fixed set of increasing
maturities $x_0,~x_1,\dots, x_n$ and we define $\tau_j$,
by $$\tau_j=x_j-x_{j-1},\quad j=1,...,N.$$ The number $\tau_j$ is
known as the {\bf tenor}. \begin{defn} We let $P_j(t)$ denote the
  discount bond price $P(t,x_j)$ and let $L_j(t)$ summarize the
  notation for a forward LIBOR rate of the type $L(t;x_{j-1},x_j)$,
  i.e.  
$$
L_j(t)=\frac{1}{\tau_j}\left(\frac{P_{j-1}(t)}{P_j(t)}-1\right)\quad
j=1,\dots,n. 
$$
\end{defn}
Recall that a \emph{vanilla cap} with \emph{cap rate} $K$ and
\emph{resettlement dates} $x_0,\dots,x_n$ is a contrat which each time
$x_j$ gives the holder of the contract the amount \begin{equation} 
\label{payoffCaplet}
h_{\gamma_j}(x_j) = \tau_j (L_j(x_{j-1})-K)^+,
\end{equation}
where $j=1,\dots,n$.
In fact, the cap is a strip of \emph{caplets}. Note that the forward
LIBOR rate $L_j(x_{j-1})$ above is in fact the
sim\-ply-\-com\-poun\-ded spot LIBOR interest rate. By definition: 
$$
L_j(x_{j-1}):=L(x_{j-1};x_{j-1},x_j)=L(x_{j-1},x_j),
$$
which is observed already at time $x_{j-1}$. The payoff
$h_{\gamma_j}$ is determined at the reset date $x_{j-1}$ but not payed
out until the settlement date $x_j$. We also note that the caplet
$\gamma_j$ is a call option on the on the underlying spot rate.

For a very long time, the market practice has been to value caps by
using a formal extension of the Black model \cite{B:1976}. This
extension is typically obtained by an approximation argument where the
short rate at one point in the argument is assumed to be
deterministic, while later on in the argument the LIBOR rate is
assumed to be stochastic. This is of course logically inconsistent. 
\begin{defn}[Black's Formula for Caplets.]
The Black-76 formula for the $j$-caplet with payoff: 
$$ 
h_{\gamma_j}(x_j)=\tau_j (L_j(x_{j-1})-K)^+,
$$
at time $t=0$ is given by the expression
\begin{equation}
\label{CapletsBlack76}
\gamma_j(h_{\gamma_j},0)= \tau_j P_j(0) \left\{L_j(0)N(d_1)-KN(d_2)
\right\}, \quad j=1,\dots,n, 
\end{equation}
where
% \begin{equation}
\begin{eqnarray}
\label{d1} d_1 & = & \displaystyle
\frac{\log\left(\frac{L_j(0)}{K}\right)+\frac{1}{2}\sigma^2_jx_j}{\sigma_j\sqrt{x_j}}   
,\\ 
\label{d2} d_2 & = & d_1-\sigma_j \sqrt{x_j}.
\end{eqnarray}
% \end{equation}
\end{defn}
The constant $\sigma_j$ is known as the \emph{Black volatility} for
the $j$-caplet, $\gamma_j$. In the market, cap prices are not quoted
in monetary terms but instead in terms of \emph{implied Black
  volatilities} and these volatilities can furthermore be quoted as
\emph{flat volatilities} or as \emph{forward rate
  volatilities} but, in this work, we confine ourselves to \emph{flat
  volatilities}. Suppose we are standing at time $t=0$ and consider
the fixed set of dates $x_0,~x_1, \dots, x_n$ where $x_0 \geq 0$, and
a fixed cap rate $K$. We assume that, for each $j=1,\dots, n$, there
is a traded cap with resettlement dates $x_0, x_1,\dots, x_j$, and we
denote the corresponding observed market price by $C^o_j$. From this
data we can easily compute the market prices for the corresponding
caplets by means of the recursion formula 
\begin{equation}
\label{CapletBootstrapping}
\gamma_j^o=C_j^o-C_{j-1}^o, \quad j=1,\dots,n
\end{equation}
with the convention $C_0^o=0$. Moreover, note that given market data
for caplets we can easily compute the corresponding market data for
caps by solving the previous recursion formula
(\ref{CapletBootstrapping}): 
\begin{equation}
\label{CapAsCapletStrip}
C_j=\displaystyle \sum_{j=1}^n \gamma_j, \quad j=1,\dots,n
\end{equation}
 Given market price data as above, the implied Black flat volatilities
 are defined as follows. 
\begin{defn}
The implied flat volatilities $\bar\sigma_1,\dots,\bar\sigma_n$ are
defined as the solutions of the equations: 
\begin{equation}
\label{FlatVols}
C^o_j = \sum_{k=1}^j \gamma^o_k(\bar\sigma_j), \quad j=1,\dots,n,
\end{equation}
\end{defn}
In other words, the flat volatility $\bar \sigma_j$ is the volatility
implied by the Black formula if you use the same volatility for each
caplet, in the cap with maturity $x_j$.
\subsection{IRS and At-The-Money Plain Vanilla Caps}
An interest rate swap (henceforth IRS) is a scheme where you exchange
a payment stream at a fixed rate of interest for a payment stream at a
floating rate (e.g. LIBOR). A payer IRS settled in arrears is
specified by: 
\begin{itemize}
\item a number of future dates $x_0 <x_1 <\dots < x_n$  where
  $x_j-x_{j-1}\equiv\tau_j$ are the settlement periods and $x_n$ is
  called the maturity of the swap, 
\item a fixed rate $K$; and,
\item a nominal value $N$.
\end{itemize}
Moreover, plain vanilla IRS satisfy the equidistance condition for the
settlement periods; i.e., $\tau \equiv \tau_j$. We recall that cash
flows take place just at the settlement dates $x_1, x_2, \dots,
x_n$. At this dates, the holder of such an IRS, pays a predetermined
amount $$K\tau N$$ and receives in turn the floating payout
$$L_j(x_{j-1})\tau N.$$ The net cash at $x_j$ is therefore 
$$
\left\{L_j(x_{j-1})-K\right\}\tau N.
$$
Without loss of generality we set the notional $N=1$ and $t=0$ with
$x_0>0$. By means of the fundamental arbitrage free pricing formula,
starting from the money-market martingale measure $\mathbb{Q}$ we can
compute the value of this contract as:

\begin{equation}
\begin{split}
\Pi_{sw}& =\sum_{j=1}^n\mathbb{E}\left[ D(0,x_j)
  \left\{L_j(x_{j-1})-K\right\}\tau \right] \\
& = \tau \left(\sum_{j=1}^n \mathbb{E}\left[D(0,x_j)L_j(x_{j-1})
  \right]-K\sum_{j=1}^n P_j(0) \right)  
\end{split}
\end{equation}
where we have used % that all the quantities $L_j(0)$ are
% $\mathcal{F}_0$-measurable and 
the well-known definition $$P_j(0)=
\mathbb{E}\left[ D(0,x_j)\right].$$ Let us consider the first term
% By Definition 6\footnote{See Chap. 1} the LIBOR forward rate $L_j(0)$
% may also be expressed as:
% $$
% L_j(0):=L(0; x_{j-1},x_j)=\frac{1}{\tau}
% \left(\frac{P(0,x_{j-1}}{x_j}-1\right)=\left(\frac{P_{j-1}(0)}{P_j(0)}-1\right)  $$ 
$$
\mathbb{E}\left[ D(0,x_j) L_j(x_{j-1}) \right].
$$
Fist of all, we may change the risk-neutral measure $\mathbb{Q}$ by
means of the \emph{Change of Numeraire Theorem} into the more suitable
$x_j$-forward measure $\mathbb{Q}^{x_j}$. Therefore, we have
\begin{equation}
\label{SwValuation}
\Pi_{sw}=\tau \left(\sum_{j=1}^n P_j(0) \mathbb{E}^{x_j} \left[ L_j(x_{j-1})
  \right] -K\sum_{j=1}^n P_j(0) \right)  
\end{equation}
The following result has a crutial role for concluding.
\begin{lema}
For every $j=1, \dots, n$, the LIBOR process $L_j(t)$ is a martingale
under the corresponding forward measure $\mathbb{Q}^{x_j}$, on the
interval $[0,x_{j-1}]$.
\end{lema}
\begin{demo}
From Definition 6\footnote{See Sect. 1.1, p. 7} for the 
sim\-ply-\-com\-poun\-ded LIBOR forward interest rate $L_j(t)$, we
have: 
$$
\tau L_j(t):=\frac{P_{j-1}(t)}{P_j(t)}-1.
$$
We recall that the process $$P_{j-1}(t)/P_j(t)$$ is the price of the
$x_{j-1}$-bond in terms of the strictly positive asset $P_j(t)$, which is, by
definition, the numeraire for the forward measure
$\mathbb{Q}^{x_j}$. The process $P_{j-1}(t)/P_j(t)$ is thus trivially
a $\mathbb{Q}^{x_j}$-martingale on the interval $[0,x_{j-1}]$, where
the normalized process is well defined. Therefore, $L_j(t)$ is also a
$\mathbb{Q}^{x_j}$-martingale on the same interval. \end{demo}  

By using the previous Lemma, we have 
$$
L_j(t)=\mathbb{E}^{x_j} \left[ L_j(s)\big| \mathcal{F}_t\right]\quad 0\leq
t\leq s\leq x_{j-1};
$$
and, in particular,
$$
L_j(0)=\mathbb{E}^{x_j} \left[ L_j(x_{j-1})\right].
$$
By substituting into (\ref{SwValuation}):
\begin{equation}
\label{SwValuationInter}
\begin{split}
\Pi_{sw} &=\tau \left(\sum_{j=1}^n P_j(0) L_j(0)-K\sum_{j=1}^n P_j(0)
\right)=\\ 
&=\tau \left[\frac{1}{\tau}\sum_{j=1}^n P_j(0)
  \left(\frac{P_{j-1}(0)}{P_j(0)}-1\right)-K\sum_{j=1}^n P_j(0)
\right]= \\
&=\left[\sum_{j=1}^n \left(P_{j-1}(0)-
    P_j(0)\right)-K\tau\sum_{j=1}^n P_j(0) \right].
\end{split}
\end{equation}
Finally, the total value $\Pi_{sw}$ at time $t=0$ is therefore
\begin{equation}
  \label{SwValuationFinal}
\Pi_{sw}=P_0(0)-P_n(0)-K\tau\sum_{j=1}^n P_j(0).
\end{equation}
\begin{propos}[General Closed-Formula for Plain Vanilla IRS.]
The total value $\Pi_{sw}(t)$ of a plain vanilla IRS settled in
arrears at time time $t\leq x_0$ and notional $N$ is
\begin{equation}
\label{SwValutaionFinalGral}
\Pi_{sw}(t)=N\left(P_0(t)-P_n(t)-K\tau\sum_{j=1}^n P_j(t)\right).
\end{equation}
\end{propos}

In contrast to the \emph{interest rate caps} pricing, which depends on
the particular choice of the volatility vector process $\sigma(t,T)$ 
within the HJM framework, the IRS closed-formula is generic. However,
note that the plain vanilla interest rate swaps remain dependent on
the term structure of discount bonds $T \mapsto P(t,T)$.
\begin{defn}[Forward Swap Rate.]
The forward swap rate (also called par swap rate) is the rate
$K_{sw}(t)$ at time $t\leq x_0$ which gives the ``fair value''
$\Pi_{sw}(t)=0$:  
$$
K_{sw}(t)=\frac{P_0(t)-P_n(t)}{\tau \sum_{j=1}^n P_j(t)}.
$$
\end{defn}
Let $t=0$ again for simplicity and suppose, as above, that $x_0>0$,
then: 
\begin{rmk}
 A plain vanilla cap is said to be at-the-money (ATM henceforth) if 
\begin{equation}
\label{ATMStrikeCap}
K=K_{sw}(0)= \frac{P_0(0)-P_n(0)}{\tau\displaystyle \sum_{j=1}^n P_j(0)}.
\end{equation}
\end{rmk}
\section{Caps under The General HJM Gaussian Model}
Let us now turn to the problem of rigorously pricing the
caplet. Remember that the payoffs on settlement dates $x_1, \dots,
x_n$ are: 
$$
h_{\gamma_j}(x_j) = \tau_j (L_j(x_{j-1})-K)^+.
$$
Note that such an stream of payoffs for the corresponding strip of caplets
$\gamma_j$, is equivalent in terms of pricing to those
$$
h_{\gamma_j}(x_{j-1})=\tau_j \mathbb{E} \left[D(x_{j-1},x_j)
  (L_j(x_{j-1})-K)^+\big| \mathcal{F}_{x_{j-1}}\right],
$$
which are received at fixing dates $x_0,\dots, x_{n-1}$. Because of
$\mathcal{F}_{x_{j-1}}$-mesurability of $L_j(x_{j-1})$, they can also
be expressed as
\begin{equation}
\begin{split}
h_{\gamma_j}(x_{j-1})&=\tau_j (L_j(x_{j-1})-K)^+ \mathbb{E}
\left[D(x_{j-1},x_j) 
  \big| \mathcal{F}_{x_{j-1}}\right]\\ &= \tau_j P_j(x_{j-1})
(L_j(x_{j-1})-K)^+. 
\end{split}
\end{equation}
From Definition 5 for the sim\-ply-\-com\-poun\-ded LIBOR spot rate
$L_j(x_{j-1})$, we know:
$$
L_j(x_{j-1}):=\frac{1}{\tau_j} \left( \frac{1}{P_j(x_{j-1})}-1\right),
$$
and after some trivial algebra \begin{equation}\begin{split}
h_{\gamma_j}(x_{j-1})& =\tau_j P_j(x_{j-1}) \left(\frac{1}{\tau_j}
  \left( \frac{1}{P_j(x_{j-1})}-1\right)-K\right)^+\\  
&=  \left(P_j(x_{j-1}) \left(\frac{1}{P_j(x_{j-1})}-1\right)-K
  \tau_jP_j(x_{j-1}) \right)^+ \\ 
&= \left(1-(1+\tau_j K)P_j(x_{j-1})\right)^+
\end{split}
\end{equation} we may finally write the following
representation for the stream of payoffs:
\begin{equation}
h_{\gamma_j}(x_{j-1})=(1+\tau_j K) \left(\kappa-P_j(x_{j-1})\right)^+
\end{equation}
where $\kappa=(1+\tau_j K)^{-1}$.

Consequently we see that a $j$-caplet is equivalent to $(1+\tau_j K)$
put options on an underlying $x_j$-bond, where the exercise date of
the option is at $x_{j-1}$ and the exercise price is $\kappa$. An
entire cap contract can thus be viewed as a portfolio of put options,
and we may use the results on Corollary 2 of Sect. 2.4 to compute the
theoretical price and, in particular, to price it under the General
Gaussian HJM model.
% \vspace*{-.7cm}
\begin{propos}[Caplet Pricing for Gaussian Forward
  Rates.] The price at $t=0$ of the $j$-caplet, $\gamma_j$, with
  payoff:
$$
h_{\gamma_j}(x_j)=(L_j(x_{j-1})-K)^+
$$
is given by
\begin{equation}
\label{GaussianForwardCaplet}
\gamma_j(h_{\gamma_j},0)=(1+\tau_j K)\left\{\kappa P_{j-1}(0)
  N(-d_-)-P_j(0) N(-d_+)\right\}
\end{equation} 
where
\begin{eqnarray}
\label{capletd+-} d_{\pm} & := & \displaystyle \frac{\log\left(\frac{P_j(0)}{\kappa 
      P_{j-1}(0)}\right)\pm\frac{1}{2}\vartheta^2(0,x_{j-1})}{\vartheta(0,x_{j-1})}\;  , \\  
\label{volmain} \vartheta^2(0,x_{j-1}) & := &\displaystyle
\int_0^{x_{j-1}} \| \varsigma(u;x_{j-1},x_j) \|^2\:du\; ;\\   
\label{volaux} \varsigma(t;x_{j-1},x_j) &  := & -\displaystyle
\int_{x_{j-1}}^{x_j} \sigma(t,s)\: ds.   
\end{eqnarray}
% \end{equation}
and $\kappa=(1+\tau_j K)^{-1}$.
\end{propos}


% \end{equation}
% and,
% \begin{equation}
% \label{GaussianForwardVol3}
