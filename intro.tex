\chapter*{Introduction}
This Ph.D. thesis is devoted to the application of dynamic consistent
families to problems arising in interest rate modelling. 

 A self contained introduction to the theoretical framework is
presented in Chapter 1. We report just those fundamental definitions
and results, like the fundamental arbitrage-free equation or the LIBOR
rate definiton among others, that are required in the following
chapters. A detailed survey on the mathematical settings of interest
rate models is presented in Chapter 2 where we present the 
Heath-Jarrow-Morton (HJM) framework for the forward rates. Finally, in
Chapter 3 we expose the standard market practice and pricing
techniques for interest rates derivatives like caps and bond options,
that will be fully developed later.

A first aim of the present work is to study consistent families of the
HJM models existing in mathematical and 
financial literature, in order to evaluate their applicability to
specific financial engineering problems like calibration or
valuation. Therefore, in Chapter 4 we review the general features 
of the geometric view of HJM models as seminally introduced by Bj\"ork
and Christensen in \cite{BC:1999}, introducing the concept of
consistent families with this class of models. 

A second part of the thesis is oriented to propose new techniques for
the application of existing consistent families. In particular, in
Chapter 5 and 6, by means of a new multiobjective extension of the
calibration techniques proposed by Herzel and Angelini in
\cite{AH:2002,AH:2005}, we develop a consistent framework for the
calibration of vanilla derivatives to consistent families.

First results obtained by the implementation of the method suggest
that this extended technique is quite robust and shows that the choice
of consistent families are really relevant in the quality of joint
calibration outcomes. At this point, it must be noted that consistency
has been in the last years one of the most important topics of
discussion in interest theory, although the lack of practical
applications up to now keep the empirical value of the whole theory
not fully comprehended. 

\thispagestyle{empty}
With a slightly different approach, in Chapter 7 we show that the
models empirically analyzed in Chapters 5 and 6 admit numerical
implementions which preserve wide open the use of the consistent families
introduced before by means of minor modifications of the standard numerical
schemes introduced in the literature. In this chapter we face one of
the most important problem in Mathematical Finance, that is the
pricing of derivative securities. We apply several discretization
and simulation techniques to the pricing of vanilla caps, the most
important derivative product in fixed income markets, bond options and
digital caps. The computational results confirms that the Crank-Nicolson
method outperforms the other numerical schemes considered, and it
encourages the search for more efficient implementations of this
specific finite difference approach. 

 It is crucial to remark that although our choice of the models
 is quite restrictive, the results seems to be good, and, from a
 theoretical and computational point of view, support the use of an
 entire consistent framework.
