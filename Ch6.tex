\chapter{Multiojective Calibration: More Empirical Evidence}

In Chapter 5 we have shown that the consistency hypothesis stated by
Bj\"ork \cite{BG:1999,BC:1999}, implies that the discount bond curve 
has to be determined at the same time as the parameters of the
model. Angelini and Herzel \cite{AH:2002, AH:2005}, originally proposed
the use of a optimization program related to the mentioned daily
calibrations, which is compatible with this joint estimation. % Perhaps, 
The milestone of this methodology is the use of an objective function
based on an error measure for just the portfolio of caps. Then, the
theoretical prices for the caps along the minimization of this measure
can be calculated at the same time that the discount bond curve is
fitted. This is an efficient method because consistent families of
discount bond curves have good analytical properties under the
Gaussian hypothesis, i.e. deterministic {\sl forward} volatility.

In last chapter, we have also provided an extension of the above
strategy which involves a multi-objective framework. As a direct
consequence, the objective function above-mentioned, is subtituted by
a {\sl scalarized} form of the intrinsic bi-objective problem. Now,
the error measure for the discount bonds is evaluated in each
iteration and could even dominate the joint optimization problem. To
this scope, we construct this {\sl scalarized} form using a convex
combination of both the cap and the bond error measures, by means of a
set of restricted weights. As a matter of fact, this approach is
richer in possible outcomes. 

%In this paper, we illustrate taking as a case study the daily
%calibrations of european caps, the relevance  in the choice of the
%initial curve. To this scope,  we compare the behaviour of consistent
%and inconsistent families in terms of model parameter stability and
%fitting capabilities. 

%As an obvious consequence, 
%to the cap prices 
% Explicacion Papel I
%In this paper, we illustrate the relevance of the choice of the initial curve in daily
%calibrations by comparing, in terms of model parameter stability and
%fitting ability, the behaviour of consistent and inconsistent
%families. Our analysis incises about the practical implications of
%consistency by means of a consistent family of functions closed to
%the usual non--consistent ones.  

% In doing so, we understand of extraordinary relevance that the joint
% calibration for the caps and bonds that encompasses consistency
% hints as we % will proof later, can be considered as a well-posed
% mathematical problem. As a matter of fact, in this paper, there is a
% clear difference between % %the optimization program which we show,
% a suitable choice for these needs, and the other one that have
% appeared in the financial literature in the % past pursuing similar
% conceptual goals. We remark that Angelini and Herzel
% \cite{AH:2002,AH:2005}, uses an approach for this optimization with
% % a different and less general objective function
% % specifications. Honestly, we do not like this point of view
% % concerning the objective program % % specifications, because in
% % doing it this way, we lose the richness of possible outcomes that
% % our original optimization program reports. This work also %shows
% % how flexible this new approach is in terms of convergence
% % properties.   

% Contestacion-Resumen Papel II

\section{Introduction}
Given the theoretical tools we have developed in the previous
chapters, we want to analyze further empirical results which support
the use of consistent families and the multi-objective calibration
techniques.

To this end, we extend such analysis to a particular humped volatility
HJM model, proposed by Ritchen and Chuang \cite{RC:1999} and Mercurio
and Moraleda \cite{MM:2000}, independently. We have chosen this model 
because it is quite popular and analytically treatable. In particular,
it provides closed formulas for Europeans interest-rate options. Moreover, 
it is a one-factor Gaussian model that seems to be more capable than the
Hull-White model for reproducing large humps in the implied cap
volatility curve. 

We perform our study by calibrating this model, first by using
simulated data and second by focusing on market a data set composed by US discount 
bonds and at-the-money flat cap volatilities quotes in two different
periods, as shown by the Figure \ref{data}.

\begin{figure}[h!]
\begin{center}

\caption{Market TSIR and TSV data in the two different market scenarios.\label{data}}
\begin{minipage}[c]{12cm}
\begin{tikzpicture}
\begin{axis}[width=12cm, xlabel={maturity (yr)}, ylabel={sample mean
    $\bar\sigma$ (\%)}]
\addplot+[error bars/.cd,y dir=both,y explicit] table 
[x=T, y=flatvolS1, y error=2sigmasS1] {VolDataStatsCh6.dat};
\addplot+[error bars/.cd,y dir=both,y explicit] table 
[x=T, y=flatvolS2, y error=2sigmasS2] {VolDataStatsCh6.dat};
\end{axis}
\end{tikzpicture}
\end{minipage}
\begin{minipage}[c]{12cm}
\begin{tikzpicture}
\begin{axis}[width=12cm, 
xlabel={maturity (yr)}, 
ylabel={sample mean TSIR (\%)},
legend style={at={(0.5,-0.2)}, anchor=north,legend columns=-1}
] 
\addplot+[error bars/.cd,y dir=both,y explicit] table 
[x=x, y=rateS1, y error=99conflevelS1] {DataStatsCh6.dat};
\addplot+[error bars/.cd,y dir=both,y explicit] table 
[x=x, y=rateS2, y error=99conflevelS2] {DataStatsCh6.dat};
\legend{Period 1, Period 2}
\end{axis}
\end{tikzpicture}
\end{minipage}

\end{center}
\end{figure}

With regard to the real market data, the first scenario depicts a
market situation where the implied cap volatility curves have a large
long-term humped shape and the term structure of interest rates is
closer to be flat. On the other hand, the second scenario has
periodically resurfaced in the market and may 
be considered more typical. In this situation the peak of the 
hump is at about the two year point. Moreover, the TSIR is not
monotonic increasing nor flat, being initially-inverted with a local
minimum at short-term maturities.

This rest of this chapter is organized as follows\footnote{This
  chapter is based on \cite{FNN:2011,FNN:2008}.}. In Section 6.2 we
give a brief overview of the model we want to consider and later we
discuss how to construct consistent families with such a
model. Section 6.3 is devoted to empirical results, first comparing
the consistent calibration with the non-consistent approach by means
of simulated data, then presenting 
the results of the fitting of the different methods with market data.
In the last section we give some final conclusions and
remarks. \section{Consistent Curves with The Model} Our work is devoted to the
one-dimensional Gaussian HJM humped 
volatility model of the form:
\begin{equation}
\label{HVHJMM}
df_t(x)=\{ \dots \}\,dt+ \left( \alpha + \beta x \right) e^{-a x}\,dW_t.
\end{equation}
Thus, the forward rate volatility function $\sigma(x)$ is
deterministic depending only on time to maturity.
% \begin{equation}
% \label{HVVol}
% \widetilde{\sigma}(x)=(\alpha+\beta x ) e^{-a x}.
% \end{equation}
Note that $\sigma(x)$ is a QE function that admits the matrix
representation (\ref{VolMatrixForm}). Therefore, $\sigma(x)$ may be
rewritten as 
\begin{eqnarray}
\label{eq:HVTSVMatrix}
\sigma(x)& = & ce^{A x} b, \text{ where } \\
\nonumber 
c& =& [\alpha~~\beta-a\alpha],\\
\nonumber
 A& =& \left[\begin{array}{cc}
0  & -a^2 \\
1  & -2a
\end{array}\right],\\
\nonumber
 b & = & \left[\begin{array}{c}
1\\
0
\end{array}\right].
\end{eqnarray}
For this model, the forward rate decomposition
$f_t(x)=\delta_t(x)+q_t(x)$, as seen before in Sect. 5.3,
eqs. (\ref{eq:Input})--(\ref{eq:ForwardDecomposition}), has
the corresponding $q_t$-process dynamics: 
\begin{eqnarray}
\label{eq:HVFactor}
dZ_t & = & A Z_t\:dt+b\:dW_t,\quad Z_0=0,\\
q_t(x) & = & C(x)Z_t,
\end{eqnarray}
with $A,~b$ as in (\ref{eq:HVTSVMatrix}) and $C(x)=c e^{A x}$. % Thus,
% (\ref{eq:Factor}) is a linear SDE in the narrow sense (see 
% Kloeden and Platen \cite{KP:1999} for details) with explicit solution
% \begin{equation}
% Z_t=\Phi_t\int^t_0 \Phi^{-1}_sb\:dW_s,
% \end{equation}
% where 
% $$
% \Phi_t=e^{A t}=e^{-a t}\left[\begin{array}{rr}
% 1+a t & -a^2 t \\
% t & 1-a t
% \end{array}\right].
% $$
Therefore, the analytical expression of the forward rate curve for the model is
given by \begin{equation}
\label{eq:HVHJMMOperational}
f_t(x)=f^o(x+t)+\frac{1}{2}\left[S^2(t+x)-S^2(x)\right]+C(x)Z_t,
\end{equation}
being $S(x)=\int_0^x \sigma(u)\, du$. After some algebraic
manipulations like using the explicit expansion for the stochastic term 
$C(x) Z_t$  
\begin{equation}
\begin{split}
c e^{A x}\left[\begin{array}{c}
Z^1_t\\
Z^2_t
\end{array}\right] & = e^{-a x}\left[\alpha~\beta-a\alpha\right]
\left[\begin{array}{rr}
1+a x & -a^2 x \\
x & 1-a x
\end{array}\right]\left[\begin{array}{c}
Z^1_t\\
Z^2_t
\end{array}\right] \\ \nonumber
& \\
& = e^{-a x} \left(\alpha  Z^1_t-a \alpha 
   Z^2_t+\beta  Z^2_t\right)+x e^{-a x}
   \left(\beta  Z^1_t-a \beta 
   Z^2_t\right),
\end{split}
\end{equation}
and expanding the deterministic term
$\frac{1}{2}\left[S^2(t+x)-S^2(x)\right]$ which is of the form 
$$
g_1(t) e^{-2 a x}+g_2(t)x e^{-2 a x}+g_3(t) x^2 e^{-2 a x}+h_1(t) e^{-a x}+h_2(t) x e^{- 2 a x},
$$  
(\ref{eq:HVHJMMOperational}) may be written as
\begin{equation}
\label{eq:HVMCexp}
\begin{split}
f_t(x) & =f^o(x+t)+ g_1(t) e^{-2 a x}+g_2(t)x e^{-2 a x}+g_3(t) x^2 e^{-2 a x}+  \\ 
&  \left(h_1(t)+\alpha  Z^1_t-a \alpha Z^2_t+\beta Z^2_t\right)e^{-a x}
+ \ \left(h_2(t)+\beta Z^1_t-a \beta Z^2_t\right) x e^{-a x}.
\end{split}
\end{equation}
Note that this formula, as its corresponding counterpart in previous
chapter (\ref{eq:MCexp}), is relevant for consistency, because it
shows which curves the model produces for a given initial curve
$f^o(x)$.
\subsection{The Minimal Consistent family} 

% To calibrate the model by means of real data, we actually need to determine the vector
% parameter $\theta=(\alpha,~\beta,~a)$. In order to estimate the forward rate volatility, 
% the statistical analysis of past data can be a possible approach, but the
% practitioners usually prefer implied volatility, laying within
% some derivative market prices, based techniques. 
% This way involves a minimization problem where 
% the loss function can be taken as
% $$
% l(\theta)=\sum^n_{i=1}(\zeta^*_i-\zeta(\theta,T_i))^2,
% $$
% where $\zeta(\theta,T_i)$ are the $i$--th theoretical derivative price maturing at time $t=T_i$, and
% $\zeta^*_i \equiv \zeta^*(T_i)$ is the $i$--th market price one.  As it is well known, see Proposition 24.15 and pages 364--366 in Bj\"ork \cite{B:2004},  the  price, at $t=0$, of the cap is given by

% % we need to know how this standard derivative contract is made. Suppose the
% %current time is $t=0$. Starting with $t=T$, we proceed backwards in steps of
% %length $\tau$ in time, constructing the the following set of times $x_i=(i+1)\tau$, with $i=0(1)n$ . Then we construct a portfolio of $n$ {\sl caplets}, struck at $K$, with
% %maturities $\{x_0,~x_1,~\dots,x_{n-1}\}$. These are called the fixing
% %dates or caplet maturity dates. The dates $\{x_1,~x_2,~\dots,~x_n\}$ are called the
% %payment dates. The cap is then just equal to this strip of {\sl caplets} and its
% %price is equal to the price of its constituent {\sl caplets}. If $\xi_j$
% %denotes the price at time $t=0$ of a {\sl caplet} with maturity date $x_j$
% %(and payment date $x_{j+1}=x_j+\tau$), then the model price of the cap is
% %given by
% %\begin{equation}
% %\label{eq:theoCap}
% %\zeta(T)=\sum_{j=0}^{n-1}\xi_j.
% %\end{equation}
% %Specifically, the payoff of each individual {\sl caplet} is 
% %$$
% %h_\xi(x_{j+1})=\tau(y_L(x_j,x_j+\tau)-K)^+.
% %$$
% %It is received at time $x_j+\tau$ and $y_L(x_j,x_j+\tau)$ is the spot
% %US--LIBOR rate prevalent over the accrual period $[x_j,x_j+\tau]$ (expressed as a
% %simple compound rate) and with fixing time $x_j$. It follows that, at time
% %$x_j$, the payoff of such a caplet is 
% %$$
% %h_\xi(x_j)=D(x_j,x_j+\tau)\tau(y_L(x_j,x_j+\tau)-K)^+.
% %$$
% %On the other hand, the price of a zero-coupon bond at time $x_j$ and maturing at
% %time 
% %bond may be expressed as 
% %$$
% %D(x_j,x_j+\tau)=(1+\tau y_L(x_j,x_j+\tau))^{-1}.
% %$$
% 	%Thus, it follows that
% %\begin{equation}
% %\begin{split}
% %\nonumber
% %h_\xi(x_j) &
% %=D(x_j,x_j+\tau)\tau\left(\frac{1}{\tau}\left[\frac{1}{D(x_j,x_j+\tau)}-1\right]-K\right)^+\\
% %& \\ 
% % & = \frac{1}{\kappa}(\kappa-D(x_j,x_j+\tau))^+.
% %\end{split}
% %\end{equation}
% %
% %A cap, therefore, may be seen as a basket of put options with maturity dates
% %$\{x_0,~x_1,~\dots,~x_{n-1}\}$ on zero-coupon bonds maturing at times
% %$\{x_1,~x_2,~\dots,~x_{n}\}$ with common strike level $\kappa=(1+\tau K)^{-1}$.    
% %This is an important fact, because our HJM model as in many other interest rate
% %models has an explicit formulae for bond option values, which means that in our case study 
% %caps can be easily priced. 
% %
% %In particular, as shown in Bj\"ork and Gombani \cite{BG:1999}, the value $\kappa
% %\cdot\xi_j$ at time $t=0$ of this a European put option with strike $\kappa$ and time to
% %maturity $x_j,$ on a bond with maturity $x_{j+1}$ have the explicit formula
% %$$
% %% \begin{equation}
% %\kappa\cdot\xi_j=\kappa D(0,x_j)N(-d_+)-D(0,x_{j+1})N(-d_-),
% %% \end{equation}
% %$$ 
% %where
% %\begin{equation}
% %\label{eq:ds}
% %d_{\pm}=\frac{\ln\frac{D(0,x_j)}{\kappa D(0,x_{j+1})}\pm
% %  \frac{1}{2}\vartheta^2(x_j)}{\vartheta(x_j)}. 
% %\end{equation}

% \begin{equation}
% \label{eq:expTheoCap}
% \zeta(T)=(1+\tau K) \left(\sum_{j=0}^{n-1}\kappa D(x_j)N(-d_+)-D(x_{j+1})N(-d_-)\right),	
% \end{equation} 
% where
% \begin{equation}
% \label{eq:ds}
% d_{\pm}=\frac{\ln\frac{D(x_j)}{\kappa D(x_{j+1})}\pm
%   \frac{1}{2}\vartheta^2(x_j)}{\vartheta(x_j)},
% \end{equation}
% the interval $[0,T]$ is subdivided with equidistant points, i.e.,
% \begin{equation}
% \label{eq:times}
% x_j=(j+1)\tau\qquad j=0,~1,~\dots,~n;
% \end{equation}
% $D(\cdot)$ is the initial discount function; and $\kappa$ equals to $(1+\tau K)^{-1}$ with $K$ denoting the {\sl cap rate}.

% The variable $\vartheta$ in (\ref{eq:ds}) is intimately related
% with the concrete multifactor Gaussian HJM model realization via the particular
% $[A,~B,~c]$ forward rate TSV selection:   
% $$
% \vartheta^2(x_j)=M(x_j) F(x_j) M'(x_j),
% $$
% where $M(x_j)$ is the matrix 
% $$
% M(x_j)=c A^{-1} \left(e^{A (x_j+\tau)}-e^{A x_j}\right),
% $$ 
% and $F(\cdot)$ satisfies
% $$
% F(\cdot)=\int_0^\cdot e^{-As} BB' e^{-A's} ds.
% $$
% Although the inversion of the matrix $A,$ the series expansion of $e^{A x},$ reveals that $M$ is not a
% singular matrix even for small values of parameter $a$. This result is also true
% for another Gaussian HJM models built from QE forward TSV families, because the
% matrix elements of $A$ are, fortunately, polynomial functions of the model
% parameters. However, due to numerical instability of the calibration process,
% when $a\to 0$, an asymptotically equivalent expression for $\vartheta$ must be
% used.

% The equations (\ref{eq:expTheoCap}) and (\ref{eq:ds}), also express the effective
% influence of {\sl ab initio} yield curve estimation on cap pricing.

% \subsection*{Consistent Curves with Gaussian Models}

% If we want to measure the actual impact that alternative choices to the
% Nelson-Siegel yield curve interpolating approach produces on derivatives pricing and hedging, we need 
% to determine consistent families for this particular model. The fundamental results
% can be found in Bj\"ork and Christensen \cite{BC:1999} in more detail. 
% We adapt some of them to our Gaussian case study without further technical discussion for the general case.
% \begin{defn} Consider the space $\mathcal{H}$ is defined as the space of all $\mathcal{C}^\infty$-functions,
% $$
% r: \mathcal{R}_+ \to \mathcal{R}
% $$
% satisfying the norm condition:
% $$
% ||r||^2=\sum_{n=0}^\infty 2^{-n} \int_0^\infty \left(\frac{d^n r}{dx^n}(x)\right)^2 e^{-\gamma x}\: dx<\infty
% $$
% where $\gamma$ is a fixed positive real number. 
% \end{defn}
% As proved by Bj\"ork and Svensson \cite{BS:2001} in Proposition 4.2, this space $\mathcal{H}$ is a Hilbert space.
% % \begin{tma}[Bj\"ork and Christensen] Consider as given the mapping
% \begin{tma} Consider as given the mapping
% $$
% G: \mathcal{Z} \to \mathcal{H}
% $$
% where the parameter space $\mathcal{Z}$ is an open connected subset of $R^d$, $\mathcal{H}$ a Hilbert space and the {\sl forward curve manifold} $\mathcal{G}\subseteq H$ is defined as $\mathcal{G}=Im(G)$.  The family $\mathcal{G}$ is consistent with the one-factor model $\mathcal{M}$ with deterministic volatility function $\sigma(\cdot)$, if and only if 
% \begin{eqnarray}
% \partial_x G(z,x)+\sigma(x)\int_0^x \sigma(s) ds & \in & {\rm
%   Im}\left[\partial_z G(z,x)\right],\\   
% \sigma(x) & \in & {\rm Im}\left[\partial_z
% G(z,x)\right],	
% \end{eqnarray}
% for all $z\in Z$.
% \end{tma}

% The statements 12 and 13 are called, respectively, \emph{the consistent drift} and
% \emph{the consistent volatility} conditions. These are easy to apply in concrete cases as shown Bj\"ork and Christensen \cite{BC:1999} or De Rossi \cite{R:2004}, among others.
% %TSV specification, $\sigma(x)=(\alpha+\beta x)e^{-a x}$, and make, for instance,
% %the following {\sl ansatz}:  
% %$$
% %G^{(0)}(z,x)=(z_1+z_2 x)e^{-a x}.
% %$$
% %Thus, ${\rm Im}\left[\partial_z G(z,x)\right]$ is spanned by the vector system
% %$S^{(0)}=\{e^{-a x},~x e^{-a x}\}$ and then, the {\sl CVC} is guaranteed. However, note that the vector 
% %\begin{equation}
% %\begin{split}
% % \partial_x G^{(0)}(z,x)+\sigma(x)\int_0^x \sigma(s) ds & = A(z) e^{-a x}+B(z)
% %xe^{-a x} + \\ \nonumber
% % & \  C e^{-2ax}+D xe^{-2ax}+Ex^2e^{-2ax},
% %\end{split}
% %\end{equation}
% %cannot be supplied by the system $S^{(0)}$. Consequently, $G(z,x)$ must
% %be modified for skipping the {\sl CDC} violation. Direct inspection shows that by
% %means of the {\sl ansatz}
% %$$
% %G^{(1)}(z,x)=G^{(0)}(z,x)+(z_3+z_4 x+z_5 x^2) e^{-2 ax},
% %$$
% %the {\sl CVC} is still preserved and now, in addition, the {\sl CDC} is also
% %satisfied\footnote{In fact, this two-steps approach needed to find $G$ in our concrete
% %case may be mimicked for generalizing the search of a minimal consistent family
% %with any kind of Gaussian multifactor HJM model $\mathcal{M}$ with QE
% %TSV.}. 
% For the particular one-factor model we consider along this work, Proposition 7.2 and 7.3 in Bj\"ork and Christensen \cite{BC:1999} may be directly applied to get the useful result:
\begin{propos} The family
\begin{equation}
\label{HMC}
G_{HMC}(z,x)=(z_1+z_2 x)e^{-a x}+(z_3+z_4 x+z_5 x^2)e^{-2 a x},
\end{equation}
is the minimal dimension consistent family with the model
characterized by deterministic volatility $\sigma(x)=(\alpha+\beta
x)e^{-a x}$.
\end{propos}
\begin{demo}
From (\ref{eq:HVMCexp}) we see that a family which is invariant under time translation is
consistent with the model if and only if it contains the linear space
$\{e^{-ax},xe^{-ax},e^{-2ax},xe^{-2ax},x^2e^{-2ax}\}$. 
\end{demo}

Similar results as discussed along the lines of Sect. 5.2.1 will turn
up over and over again, so we list some concluding remarks which the
reader may immediately derive.
\begin{lema} The following hold for the humped volatility
  Heath-Jarrow-Morton model characterized by $\sigma(x)=(\alpha +
  \beta x) e^{-a x}$. 
\begin{itemize}
\item The NS 
$$
G_{NS}(z,x)=z_1+z_2e^{-z_4 x}+z_3 x e^{-z_4 x},
$$
is not consistent with this model.
\item The family 
$$
G_{HMC}(z,x)=(z_1+z_2 x)e^{-a x}+(z_3+z_4 x+z_5 x^2)e^{-2 a x},
$$
it is the lowest dimension family consistent with the model (hereafter HMC).
\item The family
$$
G_{ANS+}(z,x)=z_1+z_2e^{-a x}+z_3 xe^{-a x}+(z_4+z_5 x+z_6 x^2)e^{-2 a x},
$$
is the simplest adjustment based on restricted NS family that allows model
consistency (hereafter ANS+).
\end{itemize}
\end{lema}

% \section{Calibration to Market Data Approaches}

% The calibration procedures can be described formally as follows. Let
% $\theta$ be the  vector $(\alpha,\beta,a)$ of parameter values for the
% model under consideration. Assume that we have time series observations of the implied
% volatilities, $\sigma^B_i$, of $N$ caps, with different ATM {\sl strikes},
% $K_i$, and maturities $T_i$ with $i=1,\dots,N,$ here $N=7.$ 
% Suppose, that at time
% $t=0$ we are also equipped with the discount function estimation, 
% $D(x)$, and that the market
% participants translate volatility quotes to cash quotes adopting 
% {\sl Black} framework. In doing so, they adopt the 
% convention that $K_i$ quantities must match forward swap rates of the interest
% rate swaps (IRS) with same reset periods that the $i$-th cap (these IRS start
% their cash flows at $t=x_0+\tau$ as the corresponding cap and have no cash value at
% $t=0$): 
% \begin{equation}
% \label{eq:Kswap}
% K_i=\frac{D(\tau)-D(T_i)}{\tau \sum_{j=1}^n D(x_j)},
% \end{equation}
% where $\tau$ is the length of the underlying caplets, and
% $x_1=2\tau,\dots,x_n=T_i$. The derivation of the formula (\ref{eq:Kswap}) can be found, for example, in Bj\"ork \cite{B:2004} (Proposition 20.7 on page 313). 

% Now, by inspection, it is clear that this market
% convention makes that $K_i$ depends on the yield-curve estimation. It allows to us
% to denote market prices of caps with
% $\zeta^*(T_i,D(x),K_i(D(x)),\sigma^B_i)$. This last
% expression emphasizes explicit and implicit dependence  (through ATM {\sl
%   strikes})  on discount function estimation even for market prices. Let
% $\zeta(T_i,D(x),K_i(D(x)),\theta)$ be the
% corresponding theoretical price under our particular model.

% \subsection*{The Two-Step Traditional Method}

% First, we choose a non-consistent parametrized family of forward
% rate curves $G(z,x)$. Let $D(z,x)$ be the zero-coupon bond prices reported by
% $G(z,x)$. Let $D^*_k$ be the corresponding discount factor observations on
% maturities $x_k$ with $k=1,\dots,M=11$. For each zero-coupon bond denoted with
% subscript $k$, the logarithmic pricing error\footnote{Recall that, for small
%   $\epsilon_k$, it is also the relative pricing error
%   $\frac{D^*_k-D(z,x_k)}{D(z,x_k)}$.} is written as follows 
% $$
% \epsilon_k(z)=\log D_k^*-\log D(z,x_k).
% $$
% Then, we have chosen in this work the sum of squared pricing errors, $SSE$, as
% objective function to minimize:
% \begin{equation}
% \label{eq:minD}
% SSE_D=\underset{z}{\textrm{min}}~\|\log D^*-\log
% D(z,x)\|^2=\underset{z}{\textrm{min}}\sum_{k=1}^M\epsilon^2_k(z).
% \end{equation}
% Now, via the least squares estimators $\hat{z},$ an entire discount factor estimation
% allows to price the caps using market practice or a HJM model. Following a
% similar scheme for the derivatives fitting to that used at the bond side we have 
% $$
% \varepsilon_i(\theta)=\log \zeta_i^*-\log \zeta(\theta,T_i).
% $$
% and
% \begin{equation}
% \label{eq:minC}
% SSE_C=\underset{\theta}{\textrm{min}}~\|\log \zeta^*-\log
% \zeta(\theta,T)\|^2=\underset{\theta}{\textrm{min}}\sum_{i=1}^N\varepsilon^2_i(\theta),
% \end{equation}
% where we have suppressed dependencies for simplicity.
% Note that yield-curve estimation is external to the model in the sense that
% there is no need to know first any of the model parameters $\theta$ for solving
% non-linear program (\ref{eq:minD}). 

% \subsection*{The Joint Calibration to Cap and Bond Prices}

%     Let us now describe in detail the joint cap-bond calibration procedure which
%     has sense in a consistent family framework. We note that in this situation the
%     parameters of the model are determined together with the initial forward
%     rate curve. This is different from the traditional fitting of HJM models, where
%     the two steps are separate, as we discussed before. 

% From expression (\ref{MCF}), we notice the dependency of the family 
% from the parameter $a$. Let $G(z,x,a)$ be a family consistent with our model
% (for instance, $G_m$ and $G_{ANS}$) and define least-squares estimators,
% $\hat{z}(a)$
% \begin{equation}
% \label{minDcons}
% \hat{z}(a)=\arg \underset{z}{\min} \sum_{k=1}^M(\log D^*_k-\log D(z,x_k,a))^2.
% \end{equation}
% From the expression 
% $$
% \log  D(z,x_k,a)=-\int_0^{x_k} G(z,s,a)\:ds=\sum_{j=1}^{n_p} M_{kj}(a) z_j,
% $$
% we note that, for consistent families and for a fixed $a,$ the problem
% (\ref{minDcons}) is linear in $z$-parameters (for the $G_m$ family $n_p=5$, and
% for the $G_{ANS}$ family $n_p=6$). Thus, $\hat{z}$ is an explicit and
% continuous function of $a$. With yield-curve estimation implemented for every
% fixed $a$, the entire discount function $D(\hat{z}(\theta),x,a)$ may be
% determined and it could be thought that the estimates $\hat{\theta}$ have to be
% found by solving the non-linear program
% \begin{equation}
% \label{eq:minCcons1}
% \begin{split}
% SSE_C = & \underset{\theta}{\min} ~\|\log
% \zeta^*\left[D(\hat{z}(\theta))\right]-\log 
% \zeta\left[D(\hat{z}(\theta),\theta,T)\right]\|^2 =\\
% = & \underset{\theta}{\min}\sum_{i=1}^N\varepsilon^2_i(\theta).
% \end{split}
% \end{equation}
% However, following the latter program we are not sure that the corresponding yield-curve
% at the minimum $\hat{\theta}$, $D(\hat{z}(\hat{\theta}),x,\hat{\theta})$, was
% the optimal value of the sequence of yield curve estimations implicit in this
% program (\ref{eq:minCcons1}). In other words, there exist reasonable doubts about the convergence
% of this algorithm because both error measures compete in general. Now, we consider the following
% decomposition for the total loss function $SSE(\theta)$
% \begin{eqnarray}
% SSE_D(\theta)&=&\|\log  D^*-M(\theta)\hat{z}(\theta)\|^2,\\
% SSE_C(\theta)&=&\|\log \zeta^*\left[D(\hat{z}(\theta))\right]-\log \zeta\left[D(\hat{z}(\theta)),\theta,T)\right]\|^2.
% \end{eqnarray}
%   Then, as an heuristic solution, we propose to modify the latter program to
%   include pricing residuals for the discount through the convex combination 
% \begin{equation}
% \label{eq:minCons2}
% SSE_\lambda  = \underset{\theta}{\min}\left((1-\lambda)\, SSE_D(\theta)+\lambda\, SSE_C(\theta)\right),
% \end{equation}
% for some fixed $\lambda\in [0,1]$.

% At this point, note that the program used by Angelini and Herzel \cite{AH:2002,AH:2005} in their works, uses a different goal attainment
% \begin{equation}
% \label{eq:minConsAH}
% SSE=\underset{\theta}{\min}\: SSE_C(\theta)
% \end{equation}
% where $SSE_C(\theta)$, and $\hat{z}(a)$  are defined trough the
% identities (\ref{eq:minCcons1}) and (\ref{minDcons}). As a
% consequence, the program used by these authors is a degenerate case of
% (\ref{eq:minCons2}) with $\lambda$ fixed equal to 1. 

% % {\bf 
% We test the robustness of this fitting algorithm for the MC family by using 1000
% extractions from three independent uniform distributions as initial guess for
% the parameters, $\theta^{(0)}$. As representative input data, $(D^*,\zeta^*)$,
% we use the sample mean along the first 75 trading dates of the second (excited)
% period under study. 
% Figure \ref{CalibTest}, shows the sample mean of the 1000 paths generated by
% the algorithm for $SSE_C(\theta^{(k)})$ and first contribution,
% $SSE_D(\theta^{(k)})$, departing from simulated $\theta^{(0)}$. After the initial
% movements on the wrong direction, first contribution corrects its behaviour for
% searching its own minimum. Moreover, the second contribution exhibits a correct
% minimization pattern. Note the slightly better results on both sides with smaller
% $\lambda$. Similar results can be obtained with the ANS family and other
% market scenarios. 

\section{Empirical Results}
We compare four different estimations of the initial discount bond
curve based on NS, HMC, ANS+ and cubic spline interpolation (hereafter
SP).

% Our first objective is to test the stability of the implicit estimation of the model
% parameters $\theta$. We consider mean, standard deviation and
% coefficient of variation of parameter estimates time series. 
% In this context the main goal is to analyze the impact that an 
% alternative interpolation
% scheme has on the fitting capabilities of the model. To this end, we use as a
% measure, the mean of the daily sum of squared errors of derivatives log prices,
% hereafter $MSE_C$. The same measure is used for the 
% zero-coupon bond prices (we denote it with
% $MSE_D$) and it is included in the analysis with the market data.

The US data set consists of 126 daily observations divided in two periods: 
first period covers from 3/7/2000 to 29/09/2000 (64 trading dates) and the second one
starts in 4/1/2001 and finish on 30/3/2001 (62 trading dates). 

% The Euro denominated set used for the analysis consists of 100 daily
% observations from 15/2/2001  
% to 4/7/2001. We point out that this Euro zone database is the same
% used in Angelini and Herzel \cite{AH:2002,AH:2005}.  
% As these authors, we divide the sample into two subperiods, Period 1 and Period 2.  
% Period 1 runs from the beginning to 19/4/2001 (46 observations) and
% it is characterized by a humped implied  
% volatility term structure. Period 2 goes from 20/4/2001 to the end
% (54 observations) and presents a decreasing implied volatility.  

With regard to the market the data set is composed of US discount bond of fourteen
maturities (1,~3,~6 and 9 months and from 1 to 10 years) and of implied
volatilities of at-the-money interest rate  caps with maturities
1,~2,~3,~4,~5,~7~and 10 years. This two windows of data comes from the same
database explained before in Chapter 5 being kindly provided by
Thomson Reuters Datastream. On the other hand, the simulated data was
obtained from 360 extractions of bond and cap prices with identical
maturities as its real-market equivalents as produced by the model
under study.

\subsection*{Simulations}
We simulate the forward rate curves of the humped volatility model at
time $t$ when initiliazed from alternative starting curves $f^o(x)$
using (\ref{eq:HVMCexp}).

% the forward curves until the time $t$ attainable by this model. % We
% % accomplished it by working out the expression (\ref{eq:HVMCexp}), and
% % writing the explicit formula for the stochastic as well as the
% % deterministic coefficients which are actually variable in time
% % evolution: the aforementioned $g_i(t)$, $Z^i_t$ and the extra ones
% % coming from initial curve translation, $f^o(x+t)$.
Next, we compute the fourteen prices of the set of discount bonds by
integrating the forward curve $f_t(x)$ in  
$$
P(t,x)= e^{-\int_0^x f_t(u)\,du},
$$
and the seven prices of the ATM caps by using equations
(\ref{eq:expTheoCap}), (\ref{ATMStrikeCap}), (\ref{capletd+-}), and 
by working out the integral (\ref{volmain}) to obtain the model implied
volatility function:
\begin{equation}
\vartheta^2(0,x_{j-1}) = \int_0^{x_{j-1}} \left[ \int_{x_{j-1}}^{x_j}
  ( \alpha+\beta(T-t) ) e^{-a(T-t)}\,dT \right]^2\,dt.
\end{equation}

 
% eq pam pam pam
% Explicacion: Funcion Vola Cheyette Model
% prices of the seven caps with formula (\ref{eq:expTheoCap}).   

The fixed model parameters, $\boldsymbol {p_0}=[ 0.002\:0.007\:0.35]^T$, have been
taken. This particular choice has similar order of magnitude as the
empirical estimations for this model reported by Angelini and Herzel
\cite{AH:2005}. As alternative starting curves, we choose HMC, ANS+
and NS fitted to the zero coupon bond prices shown in Figure
\ref{ZeroCoupon}.  
% 1. 

\begin{figure}[h!]
\caption{Discrete data for initial yield-curve estimation.\label{ZeroCoupon}} 
\begin{center}
\begin{tabular}{r|ccccccc}
\hline\hline
{\sc Maturity, $x$} & 0.083 & 0.25 & 1 & 2 & 3 & 4\\
{\sc Discount Bond, $P^o(x)$} & 0.9962 & 0.9886 & 0.9538 & 0.9069 & 0.8602 & 0.8142\\
\hline
{\sc Maturity, $x$}& 5& 6& 7& 8& 9 & 10\\
{\sc Discount Bond, $P^o(x)$}  &  0.7693 & 0.7260 & 0.6843 & 0.6445 &
0.6066 & 0.5706 \\
\hline
\end{tabular}
\end{center}
\end{figure}

Starting from the initial fitted curves, which may be denoted with
$f^o_{HMC}(x)$, $f^o_{ANS+}(x)$ and $f^o_{NS}(x)$, and according to
(\ref{eq:HVHJMMOperational}), the corresponding three different model
evolutions are calibrated to HMC, ANS+ and NS, restricting the palette
of the possible Pareto-front approximants produced by the {\sl scalarized} program
(\ref{finalMOO}), to $\omega_1=1$. In order to make calibration
results more comparable, Monte Carlo simulations are built in from
the identical random sequence $(Z^1_t,~Z^2_t)$ in all three cases. 

Following the expression (\ref{eq:HVMCexp}), it is easy to observe
that there are two consistent families, $G_{HMC}$ and $G_{ANS+}$, for
the first simulation E1, just one, $G_{ANS+}$, for the second
simulation E2, and no one for the last simulation E3. 

Figure \ref{SimResults} shows main consequences of the theory when the
model  is the {\sl truth} model. Notice that perfect calibration just
occurs, although model parameters 
are fixed {\sl a priori},  when the used family to perform
calibrations is consistent with all the future forward curves
generated from initial curve $f^o(x)$. This fact explains, for
instance, the bad performance for the NS family even on E3
experiment. Indeed, as may be seen in Figure \ref{sim2}, an incorrect
discount bond choice selection produces parameter instability and
imprecision.

\begin{figure}[h!]
\caption{Summary statistics for calibration results with simulated data.
\label{SimResults}} 
\begin{center}
\begin{tabular}{|c|c|lll|}
\hline\hline
& & HMC & ANS+ & NS \\
\hline 
 & $\varepsilon_r(\alpha)$ & 0 & 0 & 0.23 \\
  & $\varepsilon_r(\beta)$ & 0 & 0 & 0.13 \\
 & $\varepsilon_r(a)$ & 0 & 0 & 8.7$~10^{-2}$ \\
E1:& $C_v(\alpha)$ & 0 & 0 & 0.18 \\
$f_0(x)=f^o_{HMC}(x)$ & $C_v(\beta)$ & 0 & 0 & 0.14 \\
& $C_v(a)$ & 0 & 0 & 9.7$~10^{-2}$\\ 
 &  $\sigma_{LS}$ & 0 & 0 & 1.9$~10^{-3}$ \\
\hline
 & $\varepsilon_r(\alpha)$ & 0.25 &  0 & 0.28 \\
  & $\varepsilon_r(\beta)$ & 0.16 & 0 & 0.16 \\
 & $\varepsilon_r(a)$ & 0.12 & 0 & 9.5$~10^{-2}$ \\
E2: & $C_v(\alpha)$ & 3.8$~10^{-2}$ & 0 & 0.117 \\
$f_0(x)=f^o_{ANS+}(x)$ & $C_v(\beta)$ & 3.9$~10^{-2}$ & 0 & 9.1$~10^{-2}$ \\
 & $C_v(a)$ & 3.2$~10^{-2}$ & 0 & 4.8$~10^{-2}$ \\
 &  $\sigma_{LS}$ &  2.6$~10^{-4}$ & 0 & 6.7$~10^{-4}$ \\
\hline
 & $\varepsilon_r(\alpha)$ & 0.313 & 2.7$~10^{-4}$ & 0.18 \\
 & $\varepsilon_r(\beta)$ & 0.20 & 2.10$~10^{-4}$ & 0.10 \\
 & $\varepsilon_r(a)$ &  0.16 & 1.6$~10^{-5}$ &  6.7$~10^{-2}$ \\
E3: & $C_v(\alpha)$ & 2.3$~10^{-2}$ & 1.4$~10^{-4}$ & 0.17 \\
$f_0(x)=f^o_{NS}(x)$ & $C_v(\beta)$ & 2.6$~10^{-2}$ & 1.0$~10^{-4}$ & 0.111 \\
 & $C_v(a)$ & 2.2$~10^{-2}$ & 8.3$~10^{-5}$ & 6.3$~10^{-2}$ \\
 &  $\sigma_{LS}$ & 3.8$~10^{-4}$ & 3.9$~10^{-9}$ & 3.5$~10^{-4}$ \\
\hline
\end{tabular}
\end{center}
Sample statistics of the calibration on simulated data. Relative errors of the
parameters estimates are expressed in absolute value. We set to $0$ table
entries with value $<10^3\cdot${\tt eps} (variable {\tt eps~}$\sim10^{-16}$
measures {\sc Matlab} internal accuracy).  
\end{figure}
 
\begin{figure}[p]
\caption{Daily estimates of parameters $a$ and $\alpha$ for data
  simulated from the model with $\alpha=0.002$ and $a=0.35$ and
  starting forward curve $f_0(x)=f^o_{ANS+}(x)$.\label{sim2}} 

\begin{center}

\begin{minipage}[r]{13cm}
\begin{tikzpicture}
\begin{axis}[
ylabel={$a$ parameter},
no markers,
width=12cm,
height=9cm,
xmin=0,
xmax=360,
xtick={0,100,200,300},
ymin=0.25,
ymax=0.4,
ytick={0.25,.3,.35,.4},
ylabel style={font=\large},
legend style={at={(0.5,-0.2)}, anchor=north,legend columns=-1},
]
\addplot [very thick,black] table [x={t}, y={HMC}] {MonteCarlo_a.txt};
\addplot [black] table [x={t}, y={ANS+}] {MonteCarlo_a.txt}; 
\addplot [densely dashed,red] table [x={t}, y={NS}] {MonteCarlo_a.txt};  
\legend{HMC, ANS+, NS}
\end{axis}
\end{tikzpicture}
\end{minipage}
\begin{minipage}[r]{13cm}
\begin{tikzpicture}
\begin{axis}[
ylabel={$\alpha$ parameter},
no markers,
width=12cm,
height=9cm,
xmin=0,
xmax=360,
xtick={0,100,200,300},
ymin=0.0015,
ymax=0.0035,
ytick={0.0015,.002,.0025,.003, 0.0035},
% yticklabel pos=right,
ylabel style={font=\large}
]
\addplot [very thick,black] table [x={t}, y={HMC}] {MonteCarlo_alpha.txt};
\addplot [black] table [x={t}, y={ANS+}] {MonteCarlo_alpha.txt}; 
\addplot [densely dashed,red] table [x={t}, y={NS}] {MonteCarlo_alpha.txt};  

\end{axis}
\end{tikzpicture}
\end{minipage}
\end{center}
\begin{flushleft}
The straight line corresponds to daily calibration results belonging
ANS+ family, the irregular black line to the MC family and the dashed
red one to the NS family. 
\end{flushleft}
\end{figure}

\subsection*{Real Data}
The main purpose of this section is to compare the performance of the
two different calibration approaches introduce in Chapter 5 along the two
different periods of real trading dates described before. Therefore, from
now on we will only consider the calibration results obtained with the
market data. For simplicity, the consistent calibrations are carried
out by means of just the lowest dimension family, the HMC family.

Concerning the real data, calibration with consistent families are
carried out by setting the weights palette $(\omega_1,~ \omega_2)$, as
defined in (\ref{weightSpectrum}), Sect. 5.5. With regard to the
consistent calibration, the {\sl scalarized} MOO program (\ref{finalMOO}) has
also been used. The table on Figure \ref{RealResults} exhibits 
the sample mean of the daily error fitting measures, namely $RPE_C$
and $RPE_B$, and the mean and the coefficient of variation of
parameter estimates.  

\begin{figure}[h!]
\caption{Summary statistics for calibration results with US data on both periods.
\label{RealResults}} 
\begin{center}
\begin{tabular}{|c|c|lll|}
\hline\hline
& & HMC & NS & SP \\
\hline 
 & $\alpha$ &  0.0093  & 0.0098 & 0.01\\
  & $\beta$ &  0.0007  & 0.0013  & 0.0008\\
 & $a$ & 0.0024 & 0.0961 & 0.064\\
{\sc Period 1}& $C_v(\alpha)$ & 0.11 & 0.05 & 0.05\\
 & $C_v(\beta)$ & 0.33 & 0.71 & 1.30\\
& $C_v(a)$ & 0.52 & 0.54 & 0.92\\
 &  $RPE_C (\%)$ & 2.2 & 2.7 & 2.75\\ 
 &  $RPE_B (\%)$ & 0.019 & 0.047 & \\
\hline
 & $\alpha$ & 0.0087 & 0.0091 & 0.0085\\
  & $\beta$ & 0.0041  & 0.0039 & 0.0052\\
 & $a$ & 0.1469 & 0.176 & 0.2129\\
{\sc Period 2} & $C_v(\alpha)$ & 0.15 & 0.1 & 0.11 \\
& $C_v(\beta)$ & 0.72 & 0.38 & 0.35\\
 & $C_v(a)$ & 0.96 & 0.3 & 0.27\\
 &  $RPE_C$ & 1.49 & 1.56 & 1.36\\
 &  $RPE_B$ & 0.031 & 0.043 &\\
\hline 
\end{tabular}
\end{center}
\end{figure}

% It can be observed that a consistent calibration approach
% based upon the $G_{ANS}$ produces generally more stable parameters on both
% periods than the consistent one using minimal family and the traditional
% involving Nelson-Siegel family. 
On the other hand, Figure \ref{RealResultsGraficos} shows in-sample
fitting time series. The HMC family under study report good in-sample 
fitting results as compared with non-consistent approaches. However,
when we look in detail at the Period 2, the families NS and HMC and
even the cubic spline based interpolants perform slightly similar with
regard to caps calibration. In fact, the non-consistent approach based
with cubic spline interpolation, marginally outperforms all the
rest. In Figure \ref{tsWeightsCh6}, we plot the daily distribution of 
the weight $\omega_1$ which performs the best calibration for caps in
both samples of data. As for the Period 1, we observe that the cap
contribution of the scalarized objective is more dominant, because of
the reason that when articulating {\sl a posteriori} preferences as
we made in previous chapter, the weights closer to the WPO
($\omega_1=1$) have the capability to reproduce better daily fits of
derivatives. As for the Period 2, the individual objectives,
$[l_1\:l_2]^T$  appear to be more cooperative relaxing the performance
of the fit results of the implied cap volatility curve. This
fact, may explain the very similar results provided for all three 
methods in the second window.  
\pgfplotstableread{HistogramsCh6.dat}{\histoweights}  
\begin{figure}[h!]
\centering
\caption{Not normalized daily empirical distribution of weights with the best $RPE_C$
  for both sample periods as produced by the multi-objective
  calibration. \label{tsWeightsCh6}} 
\begin{tikzpicture}
  \begin{axis}[
    ybar,
    ylabel={$\#$},
    height=9cm,
    width=12cm,
    enlarge y limits=false,
    axis lines*=left,
    ymin=0,
    ymax=15,
    legend style={at={(0.5,-0.2)},
      anchor=north,legend columns=-1},
    xtick=data,
    ytick={5,10,15,20}
%   nodes near coords% ,
%   every node near coord/.append style= {
%   anchor=mid west,
%   rotate=70
%    }
  ]
    \addplot table [x={w_1}, y={countS1}] {\histoweights};
    \addplot table [x={w_1}, y={countS2}] {\histoweights};
    \legend{Period 1, Period 2}
\end{axis}
\end{tikzpicture}
\end{figure}
% Focusing on the the Euro market, we restrict ourselves to the
% comparison of three different estimations of the initial yield curve
% based on the minimal dimension family which is consistent with the
% model analysed in the paper. The table in Figure \ref{EuroRealResults}
% compares the results reported by Angelini and Herzel
% \cite{AH:2002,AH:2005} (left column) with two of the possible extra
% outcomes that our extension may produce (central and right
% column). Recall that the objective function of their works is a
% particular case of the extension presented along the paper, whenever
% the fixed parameter, $\lambda$, is fixed to the value 1. 

% As can be seen, we stress that the results for derivatives
% calibration, outperform those provided by the above authors in their
% works. As for the estimation of the discount function, in-sample mean
% statistics are marginally worse only in the second period and
% preserving the same order of magnitude. Thus, in both periods and for
% the same Euro database, we can conclude that our proposed extension
% improves clearly non consistent methodologies that are traditionally
% carried out by the practitioners\footnote{At this point, we must to
%   note the reader that our results for the Nelson and Siegel family
%   are omitted for shortness, but they are very close to the reported
%   in \cite{AH:2005} and available under request}. 
%Euro-market results reported in Angelini and Herzel \cite{AH:2005} for the same
%model, shows a slightly better performance for the minimal consistent family
%when compared against Nelson-Siegel family, even on cap prices. % In fact, we
%% reproduce better in-sample fit results with exactly the same Datastream Euro
%% database, using approach sketched in (\ref{eq:minCons2}) with
%
%% $\lambda\neq 1$. % This could be explained by the differences on contractual
%% features for the US caps combined with a global worst yield-curve
%% determination  % for the US-market.
%Euro caps are structured with 6m underlying caplets (US at-the-money caps are
%composed by 3m caplets), and then, a Euro-market cap portfolio is less sensible
%to the short-term fitting produced by an arbitrary family and, thus, to a
%particular short-term asymptotic behaviour that yield-curve market data often
%requires.      
\section{Conclusions}
In this chapter, we analyze two new consistent families of curves (the
HMC and ANS+ families) for comparing to another non-consistent approaches like
cubic spline interpolation as well as the well-known Nelson and Siegel family. In
so doing, we have tried to support and extend to another treatable
Gaussian model the empirical findings of Chapter 6.

When using simulated data it is very clear that the consistent
families for the E1 ans E2 experiments performs much better than the
non-consistent ones. Moreover, Nelson-Siegel family does not work even
if it is chosen as the starting yield-curve (recall E3
experiment). These empirical facts constitute a nice demonstration of
the theory introduced in Chapter 4, in the sense that even on absence
of {\sl model risk} only when consistent families are used, perfect
calibration may occur.

Translation of these consequences to real data is less clear, due to
{\sl model risk} and quality of data but we can infer the following
concluding remarks. In this case, the introduction of a sufficiently
rich {\sl consistent families} like HMC, which is well motivated
theoretically by Bj\"ork et al., improves in-sample fitting
capabilities on caps on bonds complementing what Angelini and Herzel
\cite{AH:2002,AH:2005} empirically found with a different set of data. 

According to the results reported for the humped volatilty model in
this chapter and the Hull-White model in Chapter 5, multi-objective
calibration would lead generally to better results in caps calibration
as compared to non-extended consistent calibration (originally
introduced in \cite{AH:2002,AH:2005}) and the 
more traditional non-consistent methodologies.

Finally, future empirical research on the matter should include
multi-factor models for capturing more appropriately the TSIR and TSV
observed in the market. % Another theoretical point regards the
% analytical study of the 
% % \newpage

% \section*{Acknowledgements}

% This work has been partially supported by grants SEJ2006-05051 and
% ECO2009-13616 from the Ministry of Science and Innovation of Spain,
% the grant PROMETEO/2008/106 from the Education Council of Generalitat
% Valenciana and  the grant PUCH-07/08 from the CEU Cardenal Herrera
% University. 



% \begin{figure}[h!]
% \caption{Discrete data for initial yield-curve estimation.\label{ZeroCoupon}} 
% \begin{center}
% \begin{tabular}{r|ccccccc}
% \hline\hline
% {\sf Maturity, $x$}& 0.25 & 1 & 2 & 3 & 4\\
% {\sf Discount Bond, $P^o(x)$}  & 0.9886 & 0.9538 & 0.9069 & 0.8602 & 0.8142\\
% \hline
% {\sf Maturity, $x$}& 5& 6& 7& 8& 9 & 10\\
% {\sf Discount Bond, $P^o(x)$}  &  0.7693 & 0.7260 & 0.6843 & 0.6445 &
% 0.6066 & 0.5706 \\
% \hline
% \end{tabular}
% \end{center}
% \end{figure}




% % % 
\begin{figure}[p]
\centering
\caption{In-sample fitting time series for the first period (left) and the
  second period (right) with real market data.\label{RealResultsGraficos}} 
\begin{minipage}[l]{7cm}
% \includegraphics[width=7cm]{EPS/FitBonos1.eps}
\begin{tikzpicture}
\begin{axis}[
ylabel={$RPE_C(\%)$},
no markers,
width=7cm,
height=7cm,
xmin=0,
xmax=65,
ymin=0.5,
ymax=3.,
ylabel style={font=\large}
]
\addplot [very thick,black] table [x={t}, y={capMC}] {tseriesCh6Sample1.dat};
\addplot [densely dashed,red] table [x={t}, y={capNS}] {tseriesCh6Sample1.dat}; 
\addplot [densely dotted,red] table [x={t}, y={capSP}] {tseriesCh6Sample1.dat}; 
\end{axis}
\end{tikzpicture}
\end{minipage}
\begin{minipage}[r]{7cm}
% \includegraphics[width=7cm]{EPS/FitBonos2.eps}
\begin{tikzpicture}
\begin{axis}[
%ylabel={$RPE_C(\%)$},
no markers,
width=7cm,
height=7cm,
xmin=0,
xmax=65,
ymin=0.5,
ymax=3.,
yticklabel pos=right
]
\addplot [very thick,black] table [x={t}, y={capMC}] {tseriesCh6Sample2.dat};
\addplot [densely dashed,red] table [x={t}, y={capNS}] {tseriesCh6Sample2.dat}; 
\addplot [densely dotted,red] table [x={t}, y={capSP}] {tseriesCh6Sample2.dat}; 
\end{axis}
\end{tikzpicture}
\end{minipage}
\begin{minipage}[l]{7cm}
% \includegraphics[width=7cm]{EPS/FitCaps1.eps}
\begin{tikzpicture}
\begin{axis}[
ylabel={$RPE_B(\%)$},
no markers,
width=7cm,
height=7cm,
xmin=0,
xmax=65,
ymin=0.,
ymax=0.3,
ylabel style={font=\large}
]
\addplot [very thick,black] table [x={t}, y={bondMC}] {tseriesCh6Sample1.dat};
\addplot [densely dashed,red] table [x={t}, y={bondNS}] {tseriesCh6Sample1.dat}; 
\end{axis}
\end{tikzpicture}
\end{minipage}
\begin{minipage}[r]{7cm}
% \includegraphics[width=7cm]{EPS/FitCaps2.eps}
\begin{tikzpicture}
\begin{axis}[
% ylabel={$RPE_B(\%)$},
no markers,
width=7cm,
height=7cm,
xmin=0,
xmax=65,
ymin=0, 
ymax=0.3,
yticklabel pos=right
]
\addplot [very thick,black] table [x={t}, y={bondMC}] {tseriesCh6Sample2.dat};
\addplot [densely dashed,red] table [x={t}, y={bondNS}] {tseriesCh6Sample2.dat}; 
\end{axis}
\end{tikzpicture}
\end{minipage}
\begin{flushleft}
The thick black line corresponds to the minimal consistent family, the
dashed red line to the NS family and the dotted one to cubic spline interpolation.
\end{flushleft}
\end{figure}
\newpage\mbox{}\thispagestyle{empty}


% EURO
% \begin{figure}[h!]
% \caption{In-sample mean statistics for calibration results with Euro data on both periods.
% \label{EuroRealResults}} 
% \begin{center}
% \begin{tabular}{|c|c|lll|}
% \hline\hline
% & & $\lambda=1$ & $\lambda=0.25$ & $\lambda=0.01$ \\
% \hline 
% {\sc Period 1}  &  $MSE_C$ & $2.3~10^{-4}$ & $2.18~10^{-4}$ & $2.19~10^{-4}$ \\ 
%   &  $MSE_D$ & $8.8~10^{-7}$ & $8.8~10^{-7}$ & $8.4~10^{-7}$\\  
% \hline
%  {\sc Period 2} &  $MSE_C$ & $3.2~10^{-4}$ & $2.7~10^{-4}$ & $2.7~10^{-4}$\\
%  &   $MSE_D$ & $6.1~10^{-7}$ & $7.0~10^{-7}$ & $6.8~10^{-7}$\\	
% \hline 
% \end{tabular}
% \end{center}
% \end{figure}
% \end{document}

% Notas:
% Interest rate option pricing with volatility humps. Review of Derivatives Research 3, 237-­262.





