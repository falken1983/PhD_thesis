\chapter{Pricing Options with a Consistent HJM Model}
\section{Introduction}
One of the main goals of financial mathematics is to determine the
prices of derivatives. In this chapter we adapt the sophisticated
machinery  introduced in literature in past years, in order to study
which pricing methods are available for interest rate derivatives in
the event that consistent families are used. 

Therefore, we will study the behaviour of some efficient numerical
implementations of the model introduced in Chapter 5, pricing the most
liquid and traded derivatives such as vanilla caps as well as
directly related deals such as bond options and binary caps. However,
to calculate prices for derivatives when models such as considered in
Chapter 6 are used, more expensive computational methods like Monte
Carlo simulation have to be used.

The rest of this chapter is arranged as follows. First, we show in
Section 7.2 how deterministic methods may be set up in a Hull-White
economy. How to implement finite difference methods for the Hull-White
model is explained in Section 7.3. Section 7.4 presents and dicusses the
computational results reported by the several lattice methods
aforementioned. Finally, basic Monte Carlo procedures for valuation
are outlined for the model analyzed in Chapter 6.

\section{Partial Differential Equation of the HW Model}
We recall the $\mathbb{Q}$-dynamics of the HW model as introduced in
Sect. 5.1: \begin{equation}
\label{HWNum}
dr(t)=\left[ \Phi(t)-a r(t)\right] dt + \sigma dW(t).
\end{equation}

A method of transforming this stochastic evolution problem into a
deterministic one is to use the well-known {\sl Feynman-Ka\v{c}} formula. 
\begin{tma}[Feynman-Ka\v c] The partial differential equation
$$
V_t + \frac{1}{2} \sigma(x,t)^2 V_{xx}+ \mu(x,t)V_x - r(x,t) V = 0
$$
with boundary conditon $h(x, T)$ has the solution
$$
V(x,t)= \mathbb{E}^\mathbb{Q} \left[ e^{-\int_t^T r(X, s) ds} h(X, T)\right],
$$
where the expectation is taken with respect to the process $X$ defined
by
$$
dX = \mu(X, t) dt+\sigma(X, t) dW.
$$
\end{tma}
\begin{demo}
A proof can be found in \cite{O:1998}. 
\end{demo}

Therefore, under an arbitrage free economy, the value of an interest
rate derivative security $V$, solves the Partial Differential Equation
(PDE, for short):
\begin{equation}
\left\{
\begin{array}{rcl}
V_t+\frac{1}{2} \sigma^2 V_{rr} +\left[ \Phi(t) - a r \right] V_r - rV &
= &
0 \\
V(r, T)& = & h(r).
\end{array}
\right.
\end{equation}
For instance, for the discount bond value we have 
$$
h(r)=1,
$$ 
and for the European-style option the corresponding payoff is
$$
h(r) = [ \phi (P(T,S)- K)] ^+
$$
at option expiry time $T$, $T<S$, where binary unit $\phi = +1$ for
the call and $\phi = -1$ for the put, as we discussed in Sect. 2.4.

\subsection{Bypassing Forward Induction}
In a recent work, Daglish \cite{D:2010} provided an extension of the
Hull and White \cite{HW:1994} approach that allows the use of implicit
methods by noting that the calculation of Arrow-Debreu prices for
interest rate securities is analogous to discretize the Forward
Fokker-Planck equation which describes the future evolution of the
transition probability densities. This approach substitutes the
traditional explicit scheme with the superior implicit Crank-Nicolson
method for approximating the aforementioned equation. 

In this section we show how the HW model we are considering
can be fitted to the initial term structure analytically which is
definitively different from both the original algorithm of Hull and
White and the above mentioned improvement in which the use of the {\sl
  forward induction} technique stands \cite{J:1991}. Later,
numerical examples confirm that this approach improves computation
times and accuracy for derivative pricing whatever the discretization
method is used, explicit or implicit.

Consider the following transformation of variables
\begin{eqnarray}
% \left\{
\label{indepChange}
\nonumber
x & = & r - \Omega(t) \\
\Omega(t) & := & e^{-a t} \left( r_0 + \int^t_0 e^{a u} \Phi(u) du \right).
% \right.
\end{eqnarray}
Let us note that $\Omega(t)$ is chosen in such a way that $x(0)=0$.
 The price of any derivative in terms of the new variable can be
 written as $w(x,x)$. We can immediately infer the following relations
 between $V$ and $w$ 
\begin{eqnarray}
\label{wDerivatives}
\nonumber
V(r,t) & \equiv & w(x,t) = w( r- \Omega(t), t) \\
V_t & = & w_t- \left(-a\Omega(t) + \Phi(t) \right) w_x \\
\nonumber
V_r & = & w_x\\
\nonumber
V_{rr} & = & w_{xx}
\end{eqnarray}
where we have used that
$$
\frac{d}{dt} \Omega(t)+a \Omega(t)=\Phi(t)
$$
Substituing this identities into (\ref{GralPPDE1}) and using $ r =
x+\Omega(t)$, the PDE reduces to 
$$
w_t+ \frac{1}{2} \sigma^2 w_{xx} - a x w_x + \left(x+\Omega(t) \right)w
= 0,
$$
where the function $\Omega(t)$ may be easily obtained by direct integration
of (\ref{indepChange}):
\begin{equation}
\label{Omega}
\Omega(t) = F(0,t) + \frac{\sigma^2}{2 a^2} \left( 1-e^{- a t} \right)^2.
\end{equation}
From now we show that with an additional transformation of function we
can make the PDE independent of this function $\Omega(t)$. If we
define the function $u(x, t)$ as
\begin{equation}
\label{udef}
u(x,t)=e^{\int_t^T \Omega(q) dq} \, w(x,t),
\end{equation}
we get the following PDE for $u$
\begin{equation}
\label{GralPPDEu}
u_t+\frac{1}{2} \sigma^2 u_{xx}-a x u_x - x u = 0
\end{equation}
and we see that this PDE has no longer coefficients dependent on
$t$. By solving (\ref{GralPPDEu}) numerically, approximate values
$U(x,t)$ for the exact solution $u(x,t)$ can be calculated. Recall
that the price $V(r,t)$ of any interest rate option is recovered from
$u(x,t)$ via
\begin{equation}
\label{Vfromu}
V(r,t)=e^{-\int_t^T \Omega(q) dq}\, u(r-\Omega(t), t),
\end{equation}
where using the analytic formula for $\Omega(t)$ given in
(\ref{Omega}), the integral of $\Omega$ can be calculated as
\begin{equation}
\label{IntOmega}
\begin{split}
\int_t^T \Omega(q) dq & = -\left( \log P(0,T) - \log P(0,t) \right) + \\
 + & \frac{\sigma^2}{2 a^3} \left( a(T-t) - 2 (e^{-at} - e^{-aT} ) +
  \frac{1}{2} (e^{-2 a t} - e^{-2 a T} ) \right),
\end{split}
\end{equation}
which allows to match the initial discount bond curve by means of a
consistent family with the model and with no need of {\sl forward
  induction}.
\section{Finite-Difference Implementation}
Let us consider a general form of the homogeneous one-dimensional parabolic
equation given by:
\begin{equation}
\label{GralPPDE1}
V_t+a(x,t)V_{xx}+b(x,t)V_x+c(x,t) V=0
\end{equation}
subject to the final condition
\begin{equation}
\label{FBC}
V(x,T)=g(x).
\end{equation}
We approximate the PDE (\ref{GralPPDE1}) in the bounded domain $[ 
x_{min}, x_{max}] \times [t_{min}, t_{max}]$. Let us define the
finite difference operator:
$$
\mathcal{L}_h V^m_n:=\frac{a^m_n}{h^2} (V^m_{n+1}-2V^m_n +
V^m_{n-1})+\frac{b^m_n}{2h} (V^m_{n+1}-V^m_{n-1}) + c^m_n V^m_n,
$$
where $l=(t_{max}-t_{min})/M$ and $h= (x_{max}-x_{min})/N$.
Our main goal is to construct an approximation, $\widehat{V}(x,t)$ of the
true solution $V(x,t)$ by using a weighted finite-difference (FD,
henceforth) scheme given by   
\begin{equation}
\label{gralweightedFD1}
\frac{\widehat{V}^{m+1}_n-\widehat{V}^m_n}{l}+(1-\nu) \mathcal{L}_h \widehat{V}^{m+1}_n+\nu\mathcal{L}_h \widehat{V}^m_n=0.
\end{equation}
As it is well-known, we recall that if:
\begin{enumerate}
\item $\nu =0$ we get the explicit finite-difference scheme,
\item $\nu =\frac{1}{2} $ we get the Crank-Nicolson implicit
  finite-difference scheme,
\item $\nu= 1$ we get the fully implicit finite-difference scheme.
\end{enumerate}
It may be shown that when $\nu = \frac{1}{2}$ the discretization error
is $O(h^2, l^2)$ whereas this error is $O(h^2, l)$ otherwise, see for instance
\cite[Chap. 8, pp. 139--156]{DHW:1995} or \cite[Chap. 2, pp. 39 and
93]{S:1985}. The implicit methods are also absolutely stable while the
explicit method has the avoidable disadvantage of certain stability
conditions as we will see later.

Applying $\mathcal{L}_h$ to the FD equation (\ref{gralweightedFD1}),
we get 
\begin{equation}
\begin{split}
0 & = \widehat{V}^{m+1}_n-\widehat{V}^m_n + \frac{(1-\nu) l
  a^{m+1}_n}{h^2}(\widehat{V}^{m+1}_{n+1}-2\widehat{V}^{m+1}_n+\widehat{V}^{m+1}_{n-1})+\\  
  & +\frac{\nu l a^m_n}{h^2}(\widehat{V}^m_{n+1}-2
  \widehat{V}^m_n+\widehat{V}^{m}_{n-1})+ 
  \frac{(1-\nu) l
    b^{m+1}_n}{2h}(\widehat{V}^{m+1}_{n+1}-\widehat{V}^{m+1}_{n-1}) +
  \\ 
& +\frac{\nu l b^{m+1}_n}{2h}(\widehat{V}^m_{n+1}-\widehat{V}^m_{n-1}) + (1-\nu) l
c^{m+1}_n \widehat{V}^{m+1}_n + \nu l c^m_n \widehat{V}^m_n. 
\end{split}
\end{equation}
Simplifying, we obtain a general form of the weighted scheme
discretization 
\begin{equation}
\label{simpleFD1}
\begin{split}
& \left(-\frac{\nu l a^m_n}{h^2}+\frac{\nu l b^m_n}{2h} \right)
\widehat{V}^m_{n-1}+  
\left( 1+\frac{2\nu l a^m_n}{h^2} - \nu l c^m_n \right) \widehat{V}^m_n +\left(-\frac{\nu l
    a^m_n}{h^2}-\frac{\nu l b^m_n}{2h} \right) \widehat{V}^m_{n+1} = \\
= & \left( \frac{(1-\nu) l a^{m+1}_n}{h^2} - \frac{(1-\nu) l b^{m+1}_n}{2
    h} \right)\widehat{V}^{m+1}_{n-1}+ \left( 1- \frac{2(1- \nu)l
    b^{m+1}_n}{h^2} +(1-\nu) l c^{m+1}_n \right) \widehat{V}^{m+1}_n +\\
+ & \left( \frac{(1-\nu) l a^{m+1}_n}{h^2} + \frac{(1-\nu) l b^{m+1}_n}{2h}
\right) \widehat{V}^{m+1}_{n+1},
\end{split}
\end{equation}
Let us introduce the coefficients:
\begin{eqnarray}
\label{gralFDcoef}
\nonumber 
A^m_n & = & \frac{l a^m_n}{h^2} - \frac{l b^m_n}{2h }\\
B^m_n & = & -\frac{2 l a^m_n}{h^2} + l c^m_n, \\
\nonumber 
C^m_n & = & \frac{l a^m_n}{h^2} + \frac{l b^m_n}{2 h}.
\end{eqnarray}
Substituting these into (\ref{simpleFD1}) we finally arrive to the more
compact:
\begin{equation}
\label{simpleFD2}
\begin{split}
& -\nu A^m_n \widehat{V}^m_{n-1}+ \left(1-\nu B^m_n \right)
\widehat{V}^m_n - \nu C^m_n \widehat{V}^m_{n+1} = \\
= & (1-\nu) A^{m+1}_n \widehat{V}^{m+1}_{n-1} + \left( 1+ (1-\nu)
B^{m+1}_n \right) \widehat{V}^{m+1}_n + (1-\nu ) C^{m+1}_n \widehat{V}^{m+1}_{n+1},
\end{split}
\end{equation}
$$
m = 0, \dots,~M.
$$

\subsection{A Stable Explicit Scheme}
% To develop a finite difference method, we need to clarify the grid
% poAints. Let $N_t$ be int
Suppose we construct a grid with steps $\Delta x = h$ symmetric along the
$x$-axis where
$$
x_{min} = -x_{max}\; , \qquad \left \lfloor \frac{N}{2} \right \rfloor  =
\left \lfloor \frac{x_{max} }{h} \right \rfloor
$$
and steps $\Delta t = l$ along the $t$-axis. A node $(m,n)$ on the
grid is a point where: 
\begin{eqnarray}
x_n & = & n h, \quad n = -\lfloor N/2 \rfloor, -\lfloor N/2 \rfloor
+1, ..., 0, 1, ..., \lfloor N/2 \rfloor -1, \lfloor N/2 \rfloor\\ 
t_m & = & m l, \quad m = 0, 1, ..., M.
\end{eqnarray}

We will next apply the following explicit scheme to the PDE
(\ref{GralPPDEu}) 
\begin{equation}
\label{EFD}
\begin{split}
0 & = U^{m+1}_n-U^m_n + \frac{ l
  a^{m+1}_n}{h^2}(U^{m+1}_{n+1}-2U^{m+1}_n+U^{m+1}_{n-1})+\\  
  & + \frac{ l b^{m+1}_n}{2h}(U^{m+1}_{n+1}-U^{m+1}_{n-1}) + l c^m_n U^m_n. 
\end{split}
\end{equation}
where we have used the notation $U^m_n$ for an approximation to $u^m_n
= u(x_n , t_m)$. So we get after solving for $U^m_n$ 
\begin{equation}
\label{EFDsolved}
U^m_n = \frac{1}{1+n h l} \left[ \left(\frac{l \sigma^2}{2 h^2} +
    \frac{1}{2} a n l\right) U^{m+1}_{n-1} + \left( -
    \frac{l\sigma^2}{h^2} + 1\right) U^{m+1}_n \left( \frac{l
      \sigma^2}{2 h^2} - \frac{1}{2} a n l \right)U^{m+1}_{n+1} \right],
\end{equation}
where we have used the identities
\begin{eqnarray}
\label{idcoef}
\nonumber
a^{m+1}_n & := & \frac{1}{2} \sigma^2\\
b^{m+1}_n & := & -n a h \\ 
\nonumber
c^m_n & := & -n h.
\end{eqnarray}
By setting $\frac{l \sigma^2}{h^2}=\frac{1}{3}$ the backward recursion 
(\ref{EFDsolved}) reduces to
\begin{equation}
\label{EFDcondensed}
U^m_n = \frac{1}{1+ nhl } \left( q_d U^{m+1}_{n-1} + q_m U^{m+1}_n +
  q_u U^{m+1}_{n+1} \right)\, ,
\end{equation} 
with 
\begin{equation}
\left\{
\begin{array}{rcl}
\nonumber
q_d & := & \frac{1}{6} + \frac{1}{2} a n l \\
\nonumber
q_m & := & \frac{2}{3} \\
\nonumber
q_u  & := & \frac{1}{6} - \frac{1}{2} a n l\; .
\end{array}
\right. 
\end{equation}

\begin{lema}[Stability Condition.] Let $u$ be the solution of the PDE (\ref{GralPPDEu}) and
  let $U$ be the solution of (\ref{EFDcondensed}). If $q_i>0$, and
  provided that $q_u + q_m + q_d = 1$, then 
$$
max \; | u^m_n - U^m_n 
  | \leq A T ( l + h^2 )   
$$
for $x_{min} \leq x_n \leq x_{max}$, and $0 \leq t_m \leq T$.
\end{lema}
\begin{demo}
It is a direct consequence of the results stated in \cite[Chap. 2,
pp. 44--45]{A:1977} and \cite[Chap. 2, pp 45--47]{S:1985} for the
canonical heat equation and its explicit discretization.
\end{demo}

For $-\frac{1}{3} \frac{1}{a l} < n < \frac{1}{3} \frac{1}{a l}$, the
numbers $q_i (n)$ are all positive, satisfying stability condition. In
order to prevent these quantities from going negative, we cannot use a
finite difference grid that is arbitrarily large. At some level $\hat{n}
< \frac{1}{3} {a l} $, we want to express $U^m_{\hat{n}}$ in terms of
$U^{m+1}_{\hat{n}}$,  $U^{m+1}_{\hat{n}-1}$ and  $U^{m+1}_{\hat{n}-2}$. By doing
so we avoid using $U^{m+1}_{\hat{n}+1}$ and the grid will remain bounded
at $\hat{n}$. If use the following approximations of the {\sl spatial}
partial derivatives 
\begin{eqnarray}
\nonumber
u_{xx} & \approx & \frac{U^{m+1}_{\hat{n}}- 2 U^{m+1}_{\hat{n}-1} +
  U^{m+1}_{\hat{n}-2}}{h ^2} \\
u_x & \approx & \frac{3 U^{m+1}_{\hat{n}} - 4 U^{m+1}_{\hat{n}-1} +
  U^{m+1}_{\hat{n}-2}}{2h}
\end{eqnarray}
we can express $U^m_{\hat{n}}$ as
\begin{equation}
\label{downEFD}
U^m_{\hat{n}} = \frac{1}{1+ nhl } \left( \hat{q}_{dd} U^{m+1}_{\hat{n}-2}
  + \hat{q}_d U^{m+1}_{\hat{n}-1} + \hat{q}_m U^{m+1}_{\hat{n}} \right)\; , 
\end{equation} 
with 
\begin{equation}
\left\{
\begin{array}{rcl}
\nonumber
\hat{q}_{dd} & := & \frac{1}{6} - \frac{1}{2} a \hat{n} l \\
\nonumber
\hat{q}_d & := & - \frac{1}{3} + 2 a \hat{n} l \\
\nonumber
\hat{q}_m  & := & \frac{7}{6} - \frac{3}{2} a \hat{n} l\; .
\end{array}
\right. 
\end{equation}
These coefficients are all positive for $\frac{1}{6} \frac{1}{a l} <
\hat{n} < \frac{1}{3} \frac{1}{a l}$. 

\begin{figure}[h!] 
\centering
\caption{Downward Branching}
\begin{picture}(100,100)(-50,-10)
\unitlength=2mm
\branchlabels mdd
\root(0,10)          0.
\tbranchdown2{$(m,\hat{n})$}        0:1,2,3.
\leaf{}{$(m+1,\hat{n})$}1.
\leaf{}{$(m+1,\hat{n}-1)$}2.
\leaf{}{$(m+1,\hat{n}-2)$}3.
\end{picture}
\end{figure}

We can analogously proceed for
bounding the grid from below imposing a level $\check{n} >
-\frac{1}{3} \frac{1}{ a l }$. At $\check{n}$ we get
\begin{equation}
\label{upFD}
U^m_{\check{n}} = \frac{1}{1+ nhl } \left( \check{q}_{uu} U^{m+1}_{\check{n}+2}
  + \check{q}_u U^{m+1}_{\check{n}+1} + \check{q}_m U^{m+1}_{\check{n}} \right)\; , 
\end{equation} 
with quantities:
\begin{equation}
\left\{
\begin{array}{rcl}
\nonumber
\check{q}_{uu} & := & \frac{1}{6} + \frac{1}{2} a \check{n} l \\
\nonumber
\check{q}_u & := & - \frac{1}{3} - 2 a \check{n} l \\
\nonumber
\check{q}_m  & := & \frac{7}{6} + \frac{3}{2} a \check{n} l\; .
\end{array}
\right. 
\end{equation}
which ara all positive for $-\frac{1}{3} \frac{1}{a l} < \check{n} <
-\frac{1}{6} \frac{1}{a l}$.

\begin{figure}[h!] 
\centering
\caption{Upward Branching}
\begin{picture}(100,100)(-50,30)
\unitlength=2mm
\branchlabels duu
\root(0,10)          0.
\tbranchup2{$(m,\hat{n})$}        0:1,2,3.
\leaf{}{$(m+1,\hat{n}+2)$}1.
\leaf{}{$(m+1,\hat{n}+1)$}2.
\leaf{}{$(m+1,\hat{n})$}3.
\end{picture}
\end{figure}

Let us note that from the definition (\ref{udef}) for exact solutions,
we may infer the identical relation for the approximants
\begin{equation}
\label{UtoW}
U^m_n \equiv U(x_n , t_m ) = e^{\int_{t_m}^T \Omega(q) dq}\, W^m_n
\end{equation}
which allows us to rewrite the differencing scheme as
$$
W^m_n = \displaystyle \frac{ e^{-\int^{(m+1)l}_{ml} \Omega(q) dq} }{1+ nhl } \left(
  q_d W^{m+1}_{n-1} + q_m W^{m+1}_n + 
  q_u W^{m+1}_{n+1} \right)\, .
$$
Note that for every $m$, the integral expression can be calculated
analitically by adapting (\ref{IntOmega}) to the grid
\begin{equation}
\label{discIntOmega}
\begin{split}
\int_{m l}^{(m+1) l} \Omega(q) dq & = -\left( \log P_{m+1}(0) - \log P_{m}(0) \right) + \\
 + & \frac{\sigma^2}{2 a^3} \left( a l - 2 ( 1 - e^{-al} )e^{-a m l} +
  \frac{1}{2} (1 - e^{-2 a l} ) e^{-2 a m l} \right), \\
 & m = 0, 1, \dots, M-1.
\end{split}
\end{equation}
We also note that the quantities 
$$
\log P_0(0)= 0,\, \log P_1(0),\, \dots,\, \log P_M(0)
$$
may be computed by means of consistent calibration, deeply analyzed
in previous chapters. With the results provided by the joint
calibration for the parameters $\boldsymbol{\hat{p}}$ and
$\hat{z}(\boldsymbol{\hat{p}})$  from a set of instrument observations 
$[P^o_1 \,\dots\, P^o_{M'}]$ and 
$[\, C^o_1 \,\dots\, C^o_{N'} \,]$, we recall that
$$
\log P_m(0)= -\int^{x_m}_0 G(z(\boldsymbol{\hat{p}}), \boldsymbol{\hat{p}}, q) dq =
\sum^{n_p}_{j=1} M_{mj} ( \boldsymbol{\hat{p}}) \hat{z}_j.
$$
It is also important to point out that for every node point $(m,n)$
the approximate price $\widehat{V}^m_n$ of any derivative is
equivalent to $W^m_n$ by construction. 

The finite difference method outlined below can be implemented as
follows. For a derivative with maturity $T$, and a given number of
steps $M$ we may calculate the step-sizes as $l = T/M$ and $ h =
\sigma \sqrt{3 l} $. Then, we set $ \hat{N} = \lceil \frac{1}{6 a l}
\rceil$ which is the first integer value on the right of the lowest
bound of $\hat{n}$ for which central and downward branching produce
positive coefficients in the backward recursions (\ref{EFDcondensed})
and (\ref{downEFD}). Therefore, reasoning by symmetry, making $\hat{n}
= -\check{n} = \hat{N}$ we can bound the grid without making this
explicit scheme unstable.   
\begin{figure}[h!] 
\centering
\caption{An example of stable explicit grid for $x$ with $ l= 10 a = 1$.} 
\begin{picture}(100,150)(60,-10)
\unitlength=2mm
\branchlabels    \,\,\,
\root(0,10)          0.
\tbranch2{$(m, n)$}        0:1,2,3.
  \tbranch2{$(m+1,n+1)$}      1:4,5,6.
    \tbranchdown2{$(\hat{N},\hat{N})$}      4:4a,4b,4c.
       \leaf{}{}           4a.
       \leaf{}{}           4b.
       \leaf{}{}           4c.
    \tbranch2{}           5:5a,5b,5c.
       \leaf{}{}           5a.
       \leaf{}{}           5b.
       \leaf{}{}           5c.
    \tbranch2{}         6:6a,6b,6c.
       \leaf{}{}           6a.
       \leaf{}{}           6b.
       \leaf{}{}           6c.
  \tbranch2{}         2:7,8,9.
    \tbranch2{}           7:7a,7b,7c.
       \leaf{}{}           7a.
       \leaf{}{}           7b.
       \leaf{}{}           7c.
    \tbranch2{}           8:8a,8b,8c.
       \leaf{}{}           8a.
       \leaf{}{}           8b.
       \leaf{}{}           8c.
    \tbranch2{}           9:9a,9b,9c.
       \leaf{}{}           9a.
       \leaf{}{}           9b.
       \leaf{}{}           9c.
  \tbranch2{}        3:10,11,12.
    \tbranch2{}        10:10a,10b,10c.
       \leaf{}{}           10a.
       \leaf{}{}           10b.
       \leaf{}{}           10c.
    \tbranch2{}        11:11a,11b,11c.
       \leaf{}{}           11a.
       \leaf{}{}           11b.
       \leaf{}{}           11c.
    \tbranchup2{$(\hat{N},-\hat{N})$}        12:12a,12b,12c.
       \leaf{}{}           12a.
       \leaf{}{}           12b.
       \leaf{}{}           12c.
\end{picture}
\begin{flushleft} For $m = 0,
  \dots, \hat{N}$ we may set up a normal branching with coefficients
  $q_u$, $q_m$, $q_d$. For $m = \hat{N}+1,\dots, \dots, M$ we build a
  bounded scheme at $ n = \hat{N}$ and $ n = -\hat{N}$.
\end{flushleft}
\end{figure}

We remark that
$$
W^M_n = h(x_n, Ml)
$$
for $ n = -\hat{N}, \dots, 0 , \dots, \hat{N}$ are the known payoff
values which allow to start the backward recursion. Finally, we recall
that due to the fact that the grid is bounded additional {\sl spatial}
boundary conditions are no needed. 
\subsection{A Crank-Nicolson   Scheme} From now, we apply the 
Crank-Nicolson implicit scheme to the PDE 
(\ref{GralPPDEu}). In this case, introducing the following ratios
$$
\varrho_1 = \frac{l}{h^2}, \; \varrho_2 = \frac{l}{h}
$$
and particularizing $\nu$ to $\frac{1}{2}$ the general weighted scheme
(\ref{simpleFD2}) may be written as 
\begin{equation}
\label{CNsimpleFDu}
% \begin{split}
% &
 -\alpha_n U^m_{n-1}+ \left(1-\beta_n \right)
U^m_n - \gamma_n U^m_{n+1} = % \\
% = & 
\alpha_n U^{m+1}_{n-1} + \left( 1+ \beta_n \right)
U^{m+1}_n + \gamma_n U^{m+1}_{n+1},
% \end{split}
\end{equation}
where
\begin{equation}
\left\{
\begin{array}{rcl}
\label{CNFDcoef}
\nonumber 
\alpha_n & = \frac{1}{2} A^m_n =  & \frac{1}{4} \sigma^2\varrho_1 +
\frac{1}{4} n a h \varrho_2 \, ,\\  
\beta_n   & = \frac{1}{2} B^m_n =  & -\frac{1}{2} \sigma^2 \varrho_1 -
\frac{1}{2} n h l \, ,\\
\nonumber 
\gamma_n & = \frac{1}{2} C^m_n = & \frac{1}{4} \sigma^2 \varrho_1 -
\frac{1}{4} n a h \varrho_2\, .  
\end{array}
\right.
\end{equation}
In this case, we consider again a symmetric {\sl spatial} domain
$[-x_{max},x_{max}]$ where the grid points $(m,n)$ are defined as
\begin{equation}
\begin{array}{rclrcl}
x_n & = & -x_{max} + n h,  &n & = & 0, 1, \dots, N\\ 
t_m & = & m l,  & m &= &0, 1, \dots, M.
\end{array}
\end{equation}
being $h = 2 x_{max} / N$ and $l = T / M$. As suggested by Cairns
\cite{C:2004} and Daglish \cite{D:2010} we fix $x_{max}$ to
$5\frac{\sigma}{2 a}$ and impose as well homogeneous Dirichlet
boundary conditions 
\begin{equation}
u(x_{max} , t_m) = u(x_{min} , t_m) = 0 \, \quad m=0,1, \dots, M.
\end{equation}
Therefore, we recall that
$$
U^m_0 = u^m_0 = 0,\; U^m_N = u^m_N = 0, \; U^M_n = h(x_n , Ml)
$$
for $m = 0,1, \dots , M$ and $n = 0, 1, \dots, N$ are known values. 

In order to clarify how to compute the approximate values for the
solution at the grid points, taking the above mentioned conditions
into account, we may break down the difference equation
(\ref{CNsimpleFDu}) as follows
% in matrix form: 
\begin{equation}
\left\{
\begin{array}{rcl}
n = 0 & \to & U^m_0  =  0 \\
n = 1 & \to & (1-\beta_1 ) U^m_1 - \gamma_1 U^m_2  =  (1+\beta_1 ) U^m_1+\gamma_1 U^m_2 \\
n = 2, \dots, N-2 & &  \textrm{Eqn.}\; (\ref{CNsimpleFDu})  \\
n = N-1 & \to & -\alpha_{N-1} U^m_{N-2} + (1-\beta_{N-1} ) U^m_{N-1}  = 
\alpha_{N-1} U^m_{N-2} + (1+\beta_{N-1} ) U^m_{N-1} \\ 
n = N & \to & U^m_N = 0
\end{array}
\right.
\end{equation}
Written in matrix form the above problem we have the linear system:
\begin{equation}
\label{CNmatrix}
D_1 \boldsymbol{U}^m = D_2 \boldsymbol{U}^{m+1}
\end{equation}
where 
\begin{equation}
  \nonumber
D_1 =\left[\begin{array}{cccccc}
1-\beta_1 & -\gamma_1 &  &  &  & \\
-\alpha_2 & 1- \beta_2 & -\gamma_2 & & & \\
& -\alpha_3 & 1- \beta_3 & -\gamma_3 & & \\
& & \ddots & \ddots & \ddots & \\
& & & -\alpha_{N-2} & 1-\beta_{N-2} & -\gamma_{N-2} \\
& & & & -\alpha_{N-1} & 1- \beta_{N-1} 
\end{array}\right],
\end{equation}

\begin{equation}
\nonumber
D_2 =\left[\begin{array}{cccccc}
1+\beta_1 & \gamma_1 &  &  &  & \\
\alpha_2 & 1+ \beta_2 & \gamma_2 & & & \\
& \alpha_3 & 1+ \beta_3 & \gamma_3 & & \\
& & \ddots & \ddots & \ddots & \\
& & & \alpha_{N-2} & 1+\beta_{N-2} & \gamma_{N-2} \\
& & & & \alpha_{N-1} & 1+ \beta_{N-1} 
\end{array}\right], 
\end{equation}
are nearly tridiagonal and
\begin{equation}
\nonumber
\boldsymbol{U}^m = \left[ U^m_1 \, U^m_2 \, \dots \, U^m_{N-1} \right]^T.
\end{equation}
Thus, provided we exactly know the terminal condition 
$$
\boldsymbol{U}^M =
\boldsymbol{u}^M = \boldsymbol{h}(Ml) = [\, h(x_1 , Ml)  \,
 \dots \, h(x_{N-1}, Ml) \,]^T
$$ 
we are able to solve this difference equation obtaining the matrix
$$
\boldsymbol{U} = \left[ \boldsymbol{U}^0 \, \boldsymbol{U}^1 \, \dots
  \, \boldsymbol{U}^M \right]  \, ,
$$
which represents the approximate solution to the matrix 
$$
\boldsymbol{u} = \left[ \boldsymbol{u}^0 \, \boldsymbol{u}^1 \, \dots
  \, \boldsymbol{u}^M \right] \, .
$$
Finally, note that if we apply the relation (\ref{UtoW}) that links
the approximants $\boldsymbol{U}^m$ and $\boldsymbol{W}^m$ next we may
to construct the following two-level difference equation: 
\begin{equation}
\label{CNmatrixW}
\begin{array}{rcll}
 D_1 \boldsymbol{W}^m & = & D_2 \boldsymbol{Z}^{m+1} & m = 0, 1,
 \dots, M-1\\
 \boldsymbol{W}^M & = & \boldsymbol{w}^M \, , & 
\end{array}
\end{equation}
where $ \boldsymbol{Z}^m = e^{-\int^{(m+1)l}_{ml} \Omega(q)
  dq}\boldsymbol{W}^m $ is known at any time stage on the above 
backward recursion formula and $\boldsymbol{w}^M = \boldsymbol{h}(Ml)$. 

\section{Numerical Examples}
The explicit finite difference (EFD henceforth) and the implicit
Crank-Nicolson (CN for short) are different from the trinomial tree
approach  of Hull and White \cite{HW:1994} (HWT hereafter), as we have
analyzed in the previous sections. 

Due to the fact that with the HWT algorithm {\sl forward induction} is
needed, this method is slower than the EFD algorithm even if we use
the Arrow-Debreu prices for partially avoiding backward recursion as
we suggested in \cite[Chap. 2, pp. 23]{N:2003} for vanilla
European-style bond options.  
\begin{figure}[h!]
\caption{Discrete data for initial discount bond estimation. All rates
  are expressed with continuous compounding. \label{zeroes}}
\begin{center}
\begin{tabular}{r|ccccccc}
\hline\hline
{\sc Maturity, $x$} & 0.083 & 0.25 & 1 & 2 & 3 & 4\\
{\sc Zero Rate, $R^o(x)$} & 3.46\% & 3.54\% & 4.02\% & 4.51\%
& 4.79\% & 4.98\% \\ 
\hline
{\sc Maturity, $x$}& 5& 6& 7& 8& 9 & 10\\
{\sc Zero Rate, $R^o(x)$}  &  5.13\% & 5.24\% & 5.35\% & 5.44\%
& 5.51\% & 5.56\%\\ 
\hline
\end{tabular}
\end{center}
\end{figure}
\subsection*{Bond Options}
As first examples, we consider the pricing of a two-year and
three-year vanilla put options, written on a five-year discount
bond. We assume parameters $a = 0.1$ and $\sigma = 0.01$, which are of
similar order of magnitude of the ones observed on the markets. In our
analysis we also use the discount curve given in the table on Figure
\ref{zeroes} which is estimated by means of the lowest dimensional
consistent family with the model introduced in (\ref{MCF}) as the MIN
family. For finite-difference EFD and CN methods, both option
prices are computed on the same grid. The bond price at time $t = S
=Ml$ is subject to the well-known condition $P^M_n = 1$. The option
price at time $ t = T = M'l$, with $M' < M$,  is subject to the
following condition  
\begin{equation}
\label{OptPayoff}
W^{M'}_n = (K - P^{M'}_n)^+.
\end{equation}
We do not used closed-form formulas for the bond price. Therefore we,
first of all, compute de bond price starting recursion form $t = S$ up
to $ t =T $. Then we apply the final condition for the option
price. The latter means to replace the bond values with option values
computed by (\ref{OptPayoff}) on the same grid $(m,n)$. Then we
compute option prices using the grid up to $t = 0$.

For the trinomial tree approach HWT, we tried to price faster
European-style options by using elementary Arrow-Debreu prices 
$Q^{M'}_n$, evaluated at the options' maturity following
\cite[Sect. 2.2.4, pp. 22--23]{N:2003}. By means of {\sl forward
  induction} and restricting the backward recursion to just computing
the bond grid values at time stages $ m = M' , \dots, M-1$ we
partially avoid the need of a full {\sl forward-then-backward} methodology. 

With regard to European-style options, the table on Figure
\ref{Puts} and Figure \ref{EuropeanPuts} confirms that both EFD and CN
algorithms are superior than HWT approach in terms of accuracy. We
note that CN algorithm converge very 
fast due that the Crank-Nicolson method is second order accuracy in
time. Convergence to within two decimals of the price in basispoints
is reached with $l\approx 0.01$. On the other hand, the EFD converges 
slightly faster than the HWT algorithm, but the difference is not very
large. 

\begin{figure}[h!]
\caption{Prices (in basispoints of the notional) for put options on 5yr
  discount bond.\label{Puts}} 
\begin{center}
\begin{tabular}{|c|c|lll|lll|}
\hline\hline
\multicolumn{2}{|c|}{} & \multicolumn{3}{|c|}{European-style} &
\multicolumn{3}{|c|}{American-style} \\
\hline
Mat. / Strike & $l$ & EFD & CN & HWT & EFD & CN & HWT  \\
\hline 
& 0.50 & 0.44 & 1.07& 1.34& 18.3& 78.6 & 18.4 \\
& 0.40 & 1.07 & 1.69& 2.11& 27.5& 78.6 & 27.6 \\
& 0.25 & 0.75 & 1.10& 1.19& 45.3& 86.0 & 45.5 \\
& 0.20 & 0.94 & 1.08& 1.32& 53.7& 85.5 & 53.8 \\
$T=2yr$ & 0.10 & 0.91 & 1.04& 1.08& 72.4& 86.7 & 72.5 \\
$K=0.78$& 0.05 & 1.00 & 1.04& 1.09& 82.9& 89.8 & 83.0 \\
& 0.04 & 1.00 & 1.04& 1.06& 85.0& 90.4 & 85.0 \\
& 0.025 & 1.03 & 1.04& 1.07& 87.8& 91.2 & 87.8 \\
& 0.02 & 1.03 & 1.04& 1.06& 88.6& 91.5 & 88.6 \\
& 0.01 & 1.04 & 1.04& 1.05& 90.2& 92.0 & 90.2 \\
\hline 
& Exact & \multicolumn{3}{|c|}{1.04} & \multicolumn{3}{|c|}{} \\
\hline
& 0.50 &  4.14& 5.01& 5.82&647&779  &648 \\
& 0.40 &  2.79& 2.89& 3.69&680&779  &680\\
& 0.25 &  4.84& 4.79& 5.65&716&779  &717 \\
& 0.20 &  4.12& 4.84& 4.73&729&779  &730  \\
$T=3yr$& 0.10 &  4.73& 4.83& 5.04&755&779  &755  \\
$K=0.85$& 0.05 &  4.67& 4.82& 4.82&767&779  &767  \\
& 0.04 &  4.80& 4.82& 4.92&769&779  &769\\
& 0.025&  4.83& 4.82& 4.90&773&779  &773 \\
& 0.02 &  4.83& 4.82& 4.89&774&779  &774 \\
& 0.01 &  4.82& 4.82& 4.85&776&779  &776\\
\hline 
& Exact & \multicolumn{3}{|c|}{4.82} & \multicolumn{3}{|c|}{} \\
\hline
\end{tabular}
\end{center}

Note: We recall that HWT refers to the Hull-White trinomial tree, and
  EFD and CN to the stable explicit and the Crank-Nicolson
  implicit algorithms, respectively. For the HWT and EFD method, we assume
  $h=\sigma \sqrt{3 l}$ as in \cite{HW:1994}. For the Crank-Nicolson
  method we set $h= \sigma \sqrt{2} l$ with boundaries at $\pm 5
  \frac{\sigma}{2 a}$ following \cite{C:2004} and  \cite{D:2010}.  

\end{figure}
As concern to time consumption performance, we report in the table on
Figure \ref{TimePuts} the calculation times needed for the three
algorithms. We see that the EFD algorithm is by far faster than the HWT
algorithm. The CN method is slower than both of them due to its more
matricial nature and because we use a LU tridiagonal solver to solve the
system. However, we remark that for achieving similar accuracies on
European-style valuation, the Crank-Nicolson may be even the fastest
method with less consumption of time for the involved calculations.
\begin{figure}[h!]
\centering
\caption{Relative pricing errors of vanilla 2yr put on 5yr discount
  bond.\label{EuropeanPuts}}
\begin{tikzpicture}
\begin{loglogaxis}[
legend style={ at={ (0.03, 0.97) }, anchor=north west},
xlabel=$l$, 
ylabel={$\varepsilon_r$}
]
\addplot table [x={step}, y={eEFD}] {BondOption25HWCh7.dat};
\addplot table [x={step}, y={eCN}] {BondOption25HWCh7.dat};
\addplot table [x={step}, y={eHWT}] {BondOption25HWCh7.dat};
\legend{EFD, CN, HWT}
\end{loglogaxis}
\end{tikzpicture}
\end{figure}

\begin{figure}[h!]
\caption{Calculation times in milliseconds, running MATLAB on an Intel
  Core 2 Duo P8600 @ 2.39GHz computer for a $2yr$ option on a $5yr$
  discount bond, with $a=0.1$ and $\sigma=0.01$.\label{TimePuts}} 
\begin{center}
\begin{tabular}{|c|rrr|rrr|}
\hline\hline
& \multicolumn{3}{|c|}{European-style} &
\multicolumn{3}{|c|}{American-style} \\ 
\hline
$l$ & EFD & CN & HWT & EFD & CN & HWT  \\
\hline 
 0.50 & 0.6 & 2.1 & 1.2  &0.97&  2.2 & 1.7 \\
 0.40 & 0.5 & 1.9 & 1.5  &0.67&  2.5 & 1.8 \\
 0.25 & 0.6 & 2.5 & 2.8  &0.86&  2.9 & 2.7 \\
 0.20 & 0.7 & 3.1 & 2.4  &0.98&  3.7 & 3.2 \\
 0.10 & 1.2 & 12.5& 5.1  &1.82&  17.7& 6.6 \\
 0.05 & 2.6 & 79.7& 11.0 &4.22&  104 & 13.6 \\
 0.04 & 3.4 & 406 & 14.1 &4.82&  401 & 17.4 \\
 0.025& 5.8 & 1246& 28.4 &8.86&  1731& 31.5 \\
 0.02 & 10.8& 2558& 36.5 &10.8&  3367& 48.1 \\
 0.01 & 22.3& 9117& 106.2&40.2& 33366& 142 \\
\hline
\end{tabular}
\end{center}
\end{figure}

While pricing a vanilla European bond option by finite differences is
certainly instructive in order to give an insight of which numerical
method may be more efficient, it is not very practical in the real
market situations because we are equipped with well-known closed-form
solutions for the HW model. Therefore, we may apply these schemes to
American options, for which exact formulas are not available. To avoid
arbitrage, the option value at each point in the grid $(m,n)$ cannot
be less than the intrinsic value (the immediate payoff if the option
is exercised). For instance, for a vanilla American-style put on a
discount bond, this means
$$
w(x,t) \geq  ( K - P(T,S) )^+, \quad t< T.
$$
From a strictly practical point of view, taking this condition into
account is not very difficult. After computing $W^m_n$, we should
check for the possibility of early exercise, and set
$$
W^m_n = {\rm max}( W^m_n , K - P^m_n).
$$
Therefore, if we want to price American-style options, we need to
construct the full grid containing the bond prices, $P^m_n$, and the
separate grid with option prices, $W^m_n$. 

Due to accuracy issues, we might prefer adopting a Crank-Nicolson
scheme. We remark that in such a case for each time layer $m$ we have
the scheme (\ref{CNmatrixW}) and we may equally compute the chance of
early excercise after the calculation of the vector $\boldsymbol{W}^m$ 
$$
\boldsymbol{W}^m = {\rm max}( \boldsymbol{W}^m, K-\boldsymbol{P}^m),
$$
allowing the implementation of this method with a LU direct solver. 

We report in the table on Figure \ref{Puts} the results for all three
methods. The results for the CN method are the best in terms of the speed
of convergence and compatible with the numerical approximations
reported by commercial black boxes such as DerivaGem or FINCAD. As can
be seen, the results for the EFD and HWT are slightly similar with
less time consumption for the stable explicit method (Figure
\ref{TimePuts}).

\subsection*{Interest Rate Caps}
Earlier in this work we have shown how the value of a vanilla cap can
be expressed as a portfolio of puts on discount bonds. 

As is explained in Sect. 3.2, the price of such a cap contract with
strike K and resettlement dates $x_0, \dots, x_{n-1}$ may be
determined wih the following representation for the stream of payoffs 
\begin{equation}
h_{\gamma_j}(x_{j-1})=(1+\tau K) \left(\kappa-P_j(x_{j-1})\right)^+
\quad j=1, \dots, n
\end{equation}
where $\kappa = (1+\tau K)^{-1}$. Therefore, the numerical problem
reduces to the valuation of the corresponding portfolio of these
European-style bond options which we have discussed in detail above.

A {\sl digital cap} is an instrument that has the same characteristics
as a vanilla cap, except that the payoff is a fixed amount paid if the
final floating rate is above the strike. Therefore, the payoff of any
digital $j$-caplet which composes it can be represented as
\begin{equation}
\label{binPayoff}
h_{\delta_j} (x_j) = \mathbbm{1}_{\{L_j(x_{j-1})-K>0 \}} \qquad j=1,\dots,n
\end{equation}
if we take as unitary the notional amount. For instance, typical
Chicago Board of Trade binary options on the target US federal funds
rate take as notional the amount of \$1000.   

By following similar algebraic manipulations as we have used in
Sect. 3.2 for the payoff of a vanilla we may numerically compute the
price of a digital cap by considering that the sequence of
payoffs at times $ x_0, \dots, x_{n-1}$,
\begin{equation}
h_{\delta_j} (x_{j-1}) = P_j(x_{j-1}) \mathbbm{1}_{ \{ P_j(x_{j-1}) -
  \kappa <0 \} } \qquad j=1,\dots,n, 
\end{equation}
is equivalent to (\ref{binPayoff}). We remark that now, $\kappa$ is
$(1+\tau K)^{-1}$ once again.

As next examples, we analyze the pricing of a five-year and ten-year
vanilla and digital caps, with semi-annual tenor. We assume the same
parameters for the HW model, $a=0.1$ and $\sigma=0.01$, as we have
used in the case of bond options. Moreover, we employ as well the MIN
family to estimate the zero rate curve from the discrete data reported
by table on Figure 7.4. In the Appendix C we derive closed-form
formulas for the digital caps under the assumptions of Gaussian
Heath-Jarrow-Morton models.
\begin{figure}[h!]
\caption{Prices of vanilla/digital European-style caps with
  semi-annual tenor. \label{Caps}} 
\begin{center}
\begin{tabular}{|c|c|lll|lll|}
\hline\hline
\multicolumn{2}{|c|}{} & \multicolumn{3}{|c|}{Vanilla} &
\multicolumn{3}{|c|}{Digital} \\
\hline
Mat. / Strike & $l$ & EFD (\%) & CN (\%) & HWT (\%) & EFD & CN & HWT  \\
\hline 
& 0.50  & 5.64& 5.50& 43.6& 7.59 &7.46& 7.49 \\
& 0.25  & 5.55& 5.50& 10.7& 7.07 &7.61& 7.03 \\
$T=10yr$& 0.1   & 5.52& 5.50& 5.56& 7.67 &7.41& 7.65 \\
& 0.05  & 5.51& 5.50& 5.52& 7.39 &7.45& 7.39 \\
$K=5.5\%$& 0.025 & 5.51& 5.50& 5.51& 7.42 &7.47& 7.41 \\
& 0.02  & 5.50& 5.50& 5.51& 7.49 &7.47& 7.49 \\
& 0.01  & 5.50& 5.50& 5.50& 7.48 &7.46& 7.48 \\
\hline 
& Exact & \multicolumn{3}{|c|}{5.50} & \multicolumn{3}{|c|}{7.46} \\
\hline
& 0.50 &3.11&3.16&41.1& 4.33&4.47&4.20 \\
& 0.25 &3.14&3.16&8.34&4.67&4.50&4.62 \\
$T=5yr$& 0.1  &3.16&3.16&3.20&4.63&4.51&4.63 \\
& 0.05 &3.16&3.16&3.17&4.69&4.51&4.68 \\
$K=5\%$& 0.025&3.16&3.16&3.17&4.52&4.53&4.52 \\
& 0.02 &3.16&3.16&3.17&4.53&4.52&4.53 \\
& 0.01 &3.16&3.16&3.16&4.51&4.52&4.51 \\
\hline 
& Exact & \multicolumn{3}{|c|}{3.16} & \multicolumn{3}{|c|}{4.52} \\
\hline 
\end{tabular}
\end{center}
\begin{flushleft}
The numerical approximations for vanilla caps are expressed in percent
of the notional amount whereas the approximants for digital caps are
presented in unitary terms. Exact valuation for binary options is
worked out in Appendix C.
\end{flushleft}
\end{figure}

From table on the Figure \ref{Caps} and Figure \ref{EuropeanCaps} we
see first that even with the stable explicit procedure described in
this work, we already get vanilla cap prices accurate to within one
basispoint taking time-steps close to the month ($l \approx 0.1$). Also
we see that the vanilla numerical approximations converge much more
quickly to the theoretical prices than the corresponding digital
approximants, maybe a direct consequence of the severe non-smooth
nature of the stream of payoffs which determine the value of these
binary options. \begin{figure}[h!]
\centering
\caption{Relative valuation errors of 10yr vanilla/digital caps with semi-annual
  tenor.\label{EuropeanCaps}} 
\begin{minipage}[l]{7cm}
\begin{tikzpicture}
\begin{loglogaxis}[
legend style={ at={ (0.03, 0.97) }, anchor=north west},
ylabel={$\varepsilon_r$},
width=7cm,
height=7cm,
ymin=0.5e-6,
ylabel style={font=\large}
]
\addplot table [x={step}, y={eEFD}] {EURCap10HWCh7.dat};
\addplot table [x={step}, y={eCN}] {EURCap10HWCh7.dat};
\addplot table [x={step}, y={eHWT}] {EURCap10HWCh7.dat};
\legend{ EFD, CN, HWT }
\end{loglogaxis}
\end{tikzpicture}
\end{minipage}
\begin{minipage}[r]{7cm}
\begin{tikzpicture}
\begin{loglogaxis}[
width=7cm,
height=7cm,
ymin=0.5e-6,
yticklabel pos=right
]
\addplot table [x={step}, y={eEFD}] {DigiCap10HWCh7.dat};
\addplot table [x={step}, y={eCN}] {DigiCap10HWCh7.dat};
\addplot table [x={step}, y={eHWT}] {DigiCap10HWCh7.dat};
\end{loglogaxis}
\end{tikzpicture}
\end{minipage}
\end{figure}

We note that in both cases, digital or plain vanilla, the implicit
Crank-Nicolson scheme converges faster than the other procedures
considered.  

\section{Monte Carlo Simulation for Consistent HJM Models}
As we have discussed above with the finite difference methods, we are
typically in front of a derivative pricing problem where we cannot
evaluate analytically the fundamental arbitrage-free
equation 
\begin{eqnarray}
\nonumber
V(h,0) & = & \mathbb{E}^\mathbb{Q} \left[ D(0,T) h(T) \right] =\\
& = & \mathbb{E}^\mathbb{Q} \left[ \exp \left( - \int_0^T
    r(u)\: du \right) h(T) \right].
\end{eqnarray}
whether we combine or not it with the Feynman-Ka\v c formula in order
to produce the corresponding valuation PDE. 

The lattice methods described in the previous sections assumed that
there was a short-rate realization for the HJM model under
consideration. When the HJM model considered is not associated to a
low dimensional markovian system being the implied short-rate process
$r$, one of the state variables, lattice-based computing times
increase very significantly and could even be impossible to
implement. Then, Monte Carlo methods offer an effective and popular
alternative to lattice methods. 

\subsection{Basic Monte Carlo}
If we recover the one-factor HJM model considered in Chapter 6
\begin{equation}
\label{HVHJM}
dF(t,T)=\{ \dots \}\,dt+ \left( \alpha + \beta (T-t) \right) e^{-a (T-t)}\,dW(t),
\end{equation}
we notice, in fact, that the spot rate process $r$ is not Markovian
since does not belong to the Ritchken and Sankarasubramanian class
\cite{RS:1992}. 

Even though the short-rate process is not Markovian, there may yet
exist a higher-dimensional Markov process having the short rate as one
of its components. At this point, we remark that the volatility
function $\sigma(t,T)$ of the model (\ref{HVHJM}) can be expressed as
a sum of separable into time and maturity dependent factors
\begin{eqnarray}
\nonumber
\sigma(t,T) & = & e^{a t} \left( (\alpha + \beta T) e^{-a T} \right)  -
t e^{a t} \left( \beta e^{-a T} \right)   = \\ 
&= &\sum_{i=1}^2 \varsigma_i(t) \varrho_i(T) .
\end{eqnarray}
Therefore as is shown by \cite[Propos. 1, pp. 4]{Ch:1996} is this case
we need $\frac{2}{2} (2+3)$ state variables to determine the forward
rate via a suitable markovian system where two of the state variables
are stochastic and describe the non-Markovian nature of the short rate
process. Thus, assuming we know an analytically treatable relation
between these stochastic variables and the spot rate process, we
finally conclude that, at best, two dimensions and time lattice-based
schemes are needed in order to approximate derivative
prices. Consequently a brief analysis of simulation techniques have
full sense for this kind of HJM model.

From now, let us take as an example the model the one-factor humped
volatility model we have analyzed in previous chapter. We remark that
the following discussion may be easily extended for any HJM model. As
we shown in (\ref{eq:HVMCexp}) we know how the forward curve produced by
the model evolves. In particular, note that the expression 
\begin{equation}
\label{eq:SRevol}
r(t) \equiv f_t(0) = f^o(t)+ g_1(t) + h_1(t)+\alpha  Z_1(t) + \left(
  \beta -a \alpha \right)  Z_2(t),
\end{equation}
describes how the spot rate evolves in time. We will assume, without
loss of generality, that, as before, we want to evaluate the price at
time $0$ of a security $V$ with maturity in time $T$.

Let us consider the following procedure:
\begin{enumerate}
\item Discretize the period $[0,T]$ into $M$ intervals of equal length
  $l = T/M$ and define $t_m = m l$.
\item For $S$ simulations denoted by $\omega$, simulate $M$
  i.i.d. standard normal random variables $\xi (t_m) \sim
  \mathcal{N}(0,1)$ for $m=1, \dots, M$. Note that the vector SDE
  (\ref{eq:HVFactor}) is linear in the narrow sense \cite{KP:1999},
  with explicit solution
\begin{equation}
\label{stochInt}
Z_t=\Phi_t\int^t_0 \Phi^{-1}_sb\:dW_s,
\end{equation}
where 
$$
\Phi_t=e^{A t}=e^{-a t}\left[\begin{array}{rr}
1+a t & -a^2 t \\
t & 1-a t
\end{array}\right].
$$
 Therefore, both $Z_{1,2}$-variables are centered Gaussian variables.
\item Calculate the simulated path $\omega$ of $r(t,\cdot)$ as
  follows; for $m=1, \dots, M$ let
\begin{equation}
\label{SRevolution}
r(t_m, \omega) = f^o(t_m)+ g_1(t_m) + h_1(t_m)+\alpha  \sqrt{v_{1,m}}
\xi (t_m, \omega) + \left( \beta -a \alpha \right)  \sqrt{v_{2,m}}
\xi (t_m, \omega)  
\end{equation}
where deterministic quantities
\begin{eqnarray}
\nonumber
v_{1,m} &= & \frac{5}{4a} \left( 1-\exp(-2at_m) \right)-\left( \frac{3}{2}t_m
+\frac{1}{2} at_m^2 \right) \exp(-2at_m),\; {\rm and;}\\
\nonumber
v_{2,m} & = & \frac{1}{4 a^3} \left(1-\exp(-2at_m) \right)- \left( \frac{1}{2a^2} t_m+
  \frac{1}{2 a} t_m^2 \right) \exp(-2at_m),
\end{eqnarray}
are derived from {\sl It\^o isometry} property of the stochastic
integral (\ref{stochInt}).
\item Evaluate $V(T,\omega) \equiv V(T, r(t_m, \omega))$ for
  simulation $\omega$. 
\item Evaluate the random discounted value of the derivative payoff
\begin{equation}
\label{randomvar}
X(\omega) = \exp \left( -\sum_{m=0}^{M-1} r(t_m, \omega) l \right)
V(T, \omega).
\end{equation}
\item Conclude with the calculation of the mean value 
$$
\bar{X} = \frac{1}{S} \sum_{\omega \in \Omega} X(\omega).
$$
which is our Monte Carlo estimate of the price.
\end{enumerate}
We point out that, by construction, this simulation procedure
naturally makes suitable the use of initial consistent families
$f^o(\cdot)$ with the model. 
As an instructional example, we consider the pricing of a one-year
European-style option on a three-year discount bond.

For evaluating the payoff 
$$
V (T, \omega) \equiv (K - P_T(x, \omega))^+
$$
we use the same strategy outlined in Sect. 6.3 under the Musiela
parameterization. In this case, we have to pay our attention in just
the two-year point, and we directly compute it by simulating forward
curves $f_T(x,\omega)$ up to $T$-time --one-year forward in this
example. Then we have to integrate them over time-to-maturity up to
$x$-point in order to determine each simulated realization $P_1(2,
\omega)$ of the discount bond composing the payoff.  

We assume the model parameters $\alpha = 0.0075$, $\beta = 0.005$ and
$a = 0.15$. We employ the HMC family as initial family $f^o(\cdot)$
for the Monte Carlo runs
(\ref{eq:HVMCexp})--(\ref{SRevolution}). Finally, we also remark that
this consistent estimation is again carried out from the discrete data
reported by table on Figure 7.4. 

From table on Figure \ref{MCPut} we see that this procedure gives rise
to two types of error: simulation error and discretization
error. First, the numer of sample paths, $S$, is finite. In fact, this means
that $\bar{X}$ is a random variable. Second, the discretization of the
period $[0, T]$ result in one more error: the approximation of the
Riemann integral with non-smooth integrand 
$$
\int_0^T r(u,\omega) \; du,
$$
by the sum in (\ref{randomvar}). We see that both errors can be
reduced to a limited extent by increasing the number of simulations
and reducing the step size $l$. We must be very careful here. We are
showing a particular and basic {\sl implementation} of simulation
approach, and could be certainly improved by reworking the simulation
based on (\ref{eq:HVMCexp})--(\ref{SRevolution}) dynamics in a more
efficient way or by enhancing the elementary discretization scheme we
have used to approximate the integral expression which involves
discounting. 
\begin{figure}[h!]
\caption{Prices (in percent of the notional) for a 1yr put option on
  3yr discount bond.\label{MCPut}} 
\begin{center}
\begin{tabular}{|c|lll|lll|}
\hline\hline
& \multicolumn{3}{|c|}{MC Estimator} &
\multicolumn{3}{|c|}{MC Standard Error} \\
\hline
$S\; /\; l$ & 0.1 & 0.05 & 0.01 & 0.1 & 0.05 & 0.01 \\
\hline 
  100 & 1.974&    2.252&    2.478&    0.18&    0.17&    0.17\\
 1000 & 2.332&    2.321&    2.315&    0.06&    0.06&    0.06\\
10000 & 2.347&    2.333&    2.346&    0.018&   0.018&  0.018\\
100000& 2.334&    2.337&    2.330&    0.006&   0.006&  0.006\\
\hline
Exact & \multicolumn{3}{|c|}{2.377} & \multicolumn{3}{|c|}{} \\
\hline 
\end{tabular}
\end{center}
\end{figure}
\newpage\mbox{}\thispagestyle{empty}
