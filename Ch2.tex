%
\chapter{Forward Rate Models}
In the nineties, Heath, Jarrow and Morton (henceforth HJM)
\cite{HJM:1992} proposed a new framework for modeling the entire
forward curve directly.
\section{The Heath-Jarrow-Morton Framework}   
The stochastic setup is as shown at the end of the Sect. 1.1.3. We
consider that there exists a unique equivalent martingale
measure $\mathbb{Q}$, associated to the money-market
account. Therefore, the $T$-bond market is complete and there are no 
arbitrage strategies. Let $W$ be a $q$-dimensional $\mathbb{Q}$-Wiener
process. 

We assume that we are given an $\mathbb{R}$-valued and
$\mathbb{R}^q$-valued stochastic process $\alpha=\alpha(\omega, t, T)$ 
and $\sigma=[~\sigma_1(\omega,t,T)~\dots~\sigma_q(\omega,t,T)~]$,
respectively, with $\alpha(\cdot,T)$ and $\sigma(\cdot,T)$
$\mathcal{F}_T$-adapted processes. We also assume that for $0\leq
t<T<\infty$, the forward rate $F(\cdot,T)$ has a stochastic
differential which under $\mathbb{Q}$ is given by
\begin{equation}
\label{eqHJM:1}
\left\{
\begin{array}{rcl}
dF(t,T)& = & \alpha(t,T) dt + \displaystyle \sum_{j=1}^q
\sigma_j(t,T)dW_j (t) \\ 
F(0,T) & = & F^o(0,T).
\end{array}
\right.
\end{equation}

Note that conceptually equation (\ref{eqHJM:1}) is one stochastic
differential in the $t$-variable for every choice of $T$. Also note 
that we use the observed forward rate curve $T \mapsto F^o(0,T)$ as
the initial condition. This will automatically give us a perfect fit
between observed and theoretical $T$-bond prices at $t=0$, thus
relieving us of the task of inverting the term structure of discount
bonds. 

% \subsection{Absence of Arbitrage}
Suppose now that we have specified $\alpha$, $\sigma$ and $\{
F^o(0,T);~T\geq 0\}$. Then we have specified the entire forward rate
structure and thus, by the relation  
\begin{equation}
\label{eqP}
P(t,T)=e^{-Z(t,T)}
\end{equation}
where
\begin{equation}
\label{eqZ}
Z(t,T)=\int_t^T F(t,s) ds
\end{equation}
we have in fact specified the entire term structure of discount bonds
$$
\{P(t,T);~T\geq 0,~0\leq t\leq T\}.
$$ 
We now show how bond price dynamics are induced by a given
specification of the forward rate dynamics. By using It\^o Lemma in 
(\ref{eqP}), we have  
\begin{equation}
\label{eqdP:1}
dP(t,T)=-P(t,T) dZ(t,T)+\frac{1}{2} P(t,T) \left( dZ(t,T)\right)^2,
\end{equation}
and it remains to compute $dZ(t,T)$. We have
$$
dZ(t,T)=d \left( \int_t^T F(t,s) ds\right)
$$
and this is a situation that is not covered by the standard It\^o
formula. Let us guess the answer.
\begin{propos}
Consider for $s \in [t_0, T]$ the It\^o process defined by 
\begin{equation}
dF(t,s) =  \alpha(t,s) dt + \sum_{j=1}^q \sigma_j(t,s)dW_j (t) 
\end{equation}
with $ t \in [t_0,s]$. Then, the dynamics for the stochastic process
(\ref{eqZ}) is
\begin{equation}
\label{eqdZ}
dZ(t,T)=\left[\left(\int_t^T \alpha(t,s) ds\right) - F(t,t) \right] dt
+ \sum_{j=1}^q \int_t^T \sigma_j(t,s) ds~dW_j (t).
\end{equation}
\end{propos} 
\begin{demo} 
See Appendix A.
\end{demo}

Therefore, by substituting in equation (\ref{eqdP:1}) 
\begin{equation}
\begin{array}{ccl}
dP(t,T) & = &
P(t,T) \left\{ \left[
    r(t)-\int_t^T\alpha(t,s)+\frac{1}{2} \sum_{j=1}^q\left(\int_t^T
      \sigma_j(t,s) ds \right)^2 \right] dt \right.\\
& - & \left. \sum^q_{j=1}\left(\int_t^T \sigma_j (t,s) ds\right)dW_j(t)\right\}
\end{array}
\end{equation}
which may be summarized as the the following:
\begin{corol}
The $\mathbb{Q}$-dynamics for the $T$-bond price, $P(t,T)$, follows
the stochastic differential equation
\begin{equation}
\label{eqdP:2}
dP(t,T)= P(t,T)
\left[\left( r(t)+A(t,T)+\frac{1}{2}\|S(t,T)\|^2 \right) dt +
  S(t,T) dW(t)\right], 
\end{equation}
where $\|\cdot\|$ denotes the Euclidean norm, and
\begin{equation}
\nonumber
\begin{array}{rcl}
A(t,T) & := & -\int_t^T\alpha(t,s)\:ds \\
S_j(t,T) & := & -\int_t^T \sigma_j(t,s)\:ds,
\end{array}
\end{equation}
and we have used the matrix notations
$$
\begin{array}{rcllll}
S(t,T) & = & [~S_1(t,T) & S_2(t,T) & \dots & S_q(t,T)~]\\
W(t) & = & [~W_1(t) & W_2(t) & \dots & W_q(t)~]^T.
\end{array}
$$
\end{corol}
\subsection{Absence of Arbitrage}
\begin{tma}[HJM Drift Condition] Assume that the family of forward
  rates is given by (\ref{eqHJM:1}) and that the induced bond market
  is arbitrage free. Under the martingale measure $\mathbb{Q}$, the
  process $\alpha$ and $\sigma$ must satisfy the following relation,
  for every $t$ and every $T\geq t$. 
\begin{equation}
\label{eqDrifCond}
\alpha(t,T)=\sigma(t,T) \int_t^T \sigma(t,s)^T\: ds.
\end{equation}
\end{tma}
\begin{demo}
Since we are modeling the market under the equivalent martingale
measure, $\mathbb{Q}$, the discounted $T$-bond price
$\widetilde{P}(t,T)$ have to be a local $\mathbb{Q}$-martingale
satisfying the following differential:
$$
d\widetilde{P}(t,T)=\widetilde{P}(t,T) S(t,T) dW(t)
$$
We now look for the corresponding differential of the ordinary
discount bond price, $P(t,T)=B(t)\widetilde{P}(t,T)$. From the It\^o
Lemma we know that
\begin{equation}
\label{BondRiskNeutral}
dP(t,T)= P(t,T) \left(r(t) dt + S(t,T) dW(t)\right),
\end{equation}
in other words, as $\mathbb{Q}$ is a martingale measure with the money
account $B$ as numeraire, the local rate of return of every asset
price under $\mathbb{Q}$ equals the short rate. We thus have
$$
A(t,T)+\frac{1}{2}\| S(t,T) \|^2=0.
$$
Taking the $T$-derivative of this equation gives us the relation
(\ref{eqDrifCond})
\end{demo}
\section{From HJM to Short-Rate Models}
What is the interplay between the short-rate dynamics and the present
HJM framework? Let us consider the simplest one-dimensional HJM model:
a constant $\sigma(t,T)\equiv \sigma>0$. Then, we have under the
risk-neutral measure $\mathbb{Q}$
$$
dF(t,T)=\sigma^2(T-t)dt + \sigma dW(t),
$$
which implies by direct integration
$$
F(t,T)=F(0,T)+\frac{\sigma^2}{2}T^2 + \sigma W(T).
$$
Hence for the short rates we obtain
$$
r(t)=F(t,t)=F(0,t)+\frac{\sigma^2}{2}t^2+\sigma W(t),
$$
and taking the differentials
$$
dr(t)=(\partial_t F(0,t)+\sigma^2 t)dt +\sigma dW(t).
$$
The observant reader may identify it with the Ho and Lee model
\cite{HL:1986}. The main inputs into the HJM framework are the forward
rate volatility processes $\sigma_j(t,T)$, and as we have shown the Ho
and Lee model is a special case of the general 1-factor HJM framework,
corresponding to a particular choice of the volatility
process. However, it has remained unclear whether other short-rate
models could be derived within the HJM framework, and whether there
exists a systematic approach for generating the short-rate models. In
general, we have the following:\begin{propos}
 Suppose that $F(0,T)$, $\alpha(t,T)$ and $\sigma(t,T)$ are
 differentiable in $T$ with $\int_0^T |\partial_u
 F(0,u)|\;du<\infty$.

Then the short-rate process is an It\^o process of the form
\begin{equation}
\label{SRMHJM}
dr(t)=\zeta(t) dt+\sigma(t,t) dW(t),
\end{equation}
where
\begin{equation}
\label{driftSRMHJM}
\zeta(t)=\alpha(t,t)+\partial_t F(0,t)+\int_0^t \partial_t
\alpha(s,t)\: ds+\int_0^t \partial_t \sigma(s,t)dW(s)
\end{equation}
\end{propos}
\begin{demo}
See Appendix A.
\end{demo}
\begin{rmk}
For every forward rate model, the arbitrage free price of a derivative
security, with $T$-payoff $h(T)$, will still be given by the general
pricing formula
$$
V(h,t)=\mathbb{E}^\mathbb{Q}\left[e^{-\int_t^T r(u)\:du} h(T)
  \big|\mathcal{F}_t\right], 
$$
where the short-rate as usual is given by $r(t)=F(t,t)$. 
\end{rmk}
\section{Forward Measures}
Equation (\ref{eq:FundEqAssetPricing}) shows to calculate the
arbitrage free price $V(t)$, of a derivative security. The value
calculated must, of course, be independent of the choice of
numeraire. Consider two numeraires $M$ and $N$ with martingale
measures $\mathbb{Q}^M$ and $\mathbb{Q}^N$. Combining the result
(\ref{eq:FundEqAssetPricing}) applied to both numeraires yields 
$$
M(t)\mathbb{E}^M\left[\frac{h(T)}{M(T)}
  \Big|\mathcal{F}_t\right]=N(t)\mathbb{E}^N\left[\frac{h(T)}{N(T)}
  \Big|\mathcal{F}_t\right] 
$$
This expression can be rewritten as
\begin{equation}
\label{eqChangeOfNumeraire:1}
\mathbb{E}^M\left[g(T) \Big|\mathcal{F}_t\right]=\mathbb{E}^N\left[
g(T)\frac{M(T)/M(t)}{N(T)/N(t)} \Big|\mathcal{F}_t\right]  
\end{equation}
where $g(T)=h(T)/M(T)$. Since, $h$, $M$ and $N$ are general, this
result holds for all random payoffs $g$ and all numeraires $M$ and
$N$.

We have now derived a way to express the expectation $g(T)$ under the
measure $\mathbb{Q}^M$ in terms of an expectation under the measure
$\mathbb{Q}^N$.
\begin{tma}[Change of Numeraire]
Let $\mathbb{Q}^M$ be the equivalent martingale measure with respect
to the numeraire $M(t)$. Let $\mathbb{Q}^N$ be the equivalent
martingale measure with respect to the numeraire $N(t)$. The
Radon-Nikodym derivative that changes the equivalent martingale
measure $\mathbb{Q}^M$ into $\mathbb{Q}^N$ is given by
$$
\lambda(t)=\frac{d\mathbb{Q}^M}{d\mathbb{Q}^N}=\frac{M(T)/M(t)}{N(T)/N(t)}.
$$
\end{tma}
The \emph{Change of Numeraire Theorem} is very powerful in the context of
pricing interest rate derivatives. Instead of using the value of the
money-market account $B(t)$ as a numeraire, the prices of $T$-bonds
can also be used as a numeraire. A very convenient choice is to use
the discount bond with maturity $T$ as a numeraire for derivatives
which have a payoff $h(T)$ at time $T$. Assume, on the other hand,
that the probability measure $\mathbb{Q}^T$ associated to the
numeraire $P(t,T)$ actually exists. Hence, we can apply the
\emph{Change of Numeraire Theorem} as follows. Under the measure
$\mathbb{Q}^T$ the prices $V(h,t)/P(t,T)$ are martingales for
$t<T$. Therefore, applying the definition of a martingale and taking
into account that $P(T,T)=1$, we obtain
\begin{equation}
\label{eq:FundEqAssetPricingForward}
V(h,t)=P(t,T)\mathbb{E}^T\left[h(T)\big|\mathcal{F}_t\right]
\end{equation}
The measure $\mathbb{Q}^T$ has another very interesting property,
which virtually gave the name $T$-forward measure. Under the
$T$-forward measure, the instantaneous forward rate, $F(t,T)$ is equal
to the expected of the spot interest rate at time $T$. In formulas
$$
F(t,T)=\mathbb{E}^T \left[ r(T)\big| \mathcal{F}_t \right],
$$
e.g., see the straightforward arguments followed by Filipovi\`c in
\cite[Sect. 7.1]{Fil:2009} or Bj\"ork in
\cite[Sect. 19.4.2]{B:2004}. Note that in this case, the corresponding
Radon-Nikodym derivative that changes the $T$-forward measure
$\mathbb{Q}^T$ into the risk-neutral measure (or money-market measure)
$\mathbb{Q}$, is 
\begin{equation}
\label{eq:ForwardTOMoneyMarket}
\frac{d\mathbb{Q}^T}{d\mathbb{Q}}=\frac{P(T,T)/P(t,T)}{B(T)/B(t)}=\frac{D(t,T)}{P(t,T)}=\frac{e^{-\int_t^T  r(s)\: ds}}{P(t,T)}.
\end{equation}

\section{The General HJM Gaussian Model}
Consider a general HJM model under the risk-neutral measure
$\mathbb{Q}$ specified by (\ref{eqHJM:1}). We also assume that 
$$
\sigma(t,T)= \left[\:\sigma_1(t,T)\:\dots\:\sigma_q(t,T)\:\right]
$$
are deterministic functions of $(t,T)$, and hence forward rates
$F(t,T)$ are Gaussian distributed. We consider now a European call
option, with expiration date $T$ and exercise price $K$, on an
underlying bond with maturity $S$ (where of course $T<S$). The
following general pricing formula may be derived:
\begin{propos}[Bond Option Pricing for Gaussian Forward Rates] The
  price, at $t=0$ of the bond option
$$
h(T)=(P(T,S)-K)^+
$$
is given by
\begin{equation} 
\label{eqGaussianFROptionsFormula:1}
V(h,0)=P(0,S)N(d_+)-K P(0,T)N(d_-)
\end{equation}
where
\begin{equation}
\label{d+-}
\begin{array}{rcl}
d_{\pm} & := & \displaystyle \frac{\log\left(\frac{P(0,S)}{K
      P(0,T)}\right)\pm\frac{1}{2}\vartheta^2(T,S)}{\vartheta(T,S)}\; , \\ 
\vartheta^2(T,S) & := &\displaystyle \int_0^T \| \varsigma(u;T,S)
\|^2\:du\; ;
\end{array}
\end{equation}
and,
\begin{equation}
\label{GaussianForwardVol2}
\varsigma(t;T,S) := S(t,S)-S(t,T)=-\int_T^S \sigma(t,s)\: ds.
\end{equation}
\end{propos}
\begin{demo}
Let us start with the fundamental arbitrage-free equation
$$
V(h,0)=\mathbb{E}\left[ D(0,T) (P(T,S)-K)^+\right],
$$
where we are taking the expectations with respect the equivalent
martingale measure $\mathbb{Q}$ associated to the money-market
numeraire $B(\cdot)$.
We decompose it as follows
\begin{equation}
\label{Decomp}
V=\mathbb{E}\left[ D(0,T) P(T,S)\mathbbm{1}_{\{P(T,S)\geq
    K\}}\right]-K \mathbb{E}\left[ D(0,T)\mathbbm{1}_{\{P(T,S)\geq
    K\}}\right] 
\end{equation}
In this case, the Radon-Nikodym derivative that changes $S$-forward
measure $\mathbb{Q}^S$ into the money-market measure $\mathbb{Q}$ will be
given by
$$
\lambda^S(T)=\frac{d\mathbb{Q}^S}{d\mathbb{Q}}=\frac{P(T,S)/P(0,S)}{B(T)/B(0)}=\frac{D(0,T)P(T,S)}{P(0,S)}.  
$$
In a similar way note that
$$
\lambda^T(T)=\frac{d\mathbb{Q}^T}{d\mathbb{Q}}=\frac{P(T,T)/P(0,T)}{B(T)/B(0)}=\frac{D(0,T)}{P(0,T)},
$$
is the conversion factor responsible for changing the $T$-forward
measure $\mathbb{Q}^T$ into the risk-neutral world
$\mathbb{Q}$. Substituting into decomposition (\ref{Decomp}), and 
combining with the measurability at $t=0$ of $P(0,S)$ and $P(0,T)$ we
have
\begin{equation} 
\nonumber
V = \mathbb{E}\left[ P(0,S) \lambda^S(T) \mathbbm{1}_{\{P(T,S)\geq
    K\}}\right]-K \mathbb{E}\left[ P(0,T) \lambda^T(T)
  \mathbbm{1}_{\{P(T,S)\geq K\}}\right] 
\end{equation}
\begin{equation} 
\nonumber
V = P(0,S) \mathbb{Q}^S\left( P(T,S)\geq K\right)-K P(0,T)
\mathbb{Q}^T\left( P(T,S)\geq K \right)
\end{equation}
Now we have the value $V$ for the call option in terms of the forward
measures $\mathbb{Q}^S$ and $\mathbb{Q}^T$. Let us start with the
probability computations referred to the $T$-forward measure. Note
that the probability may be written as
$$
\mathbb{Q}^T(P(T,S)\geq K)=\mathbb{Q}^T\left(\frac{P(T,S)}{P(T,T)}\geq K\right)
=\mathbb{Q}^T\left(\log\frac{P(T,S)}{P(T,T)}\geq \log K\right)$$
Consider the ``discounted'' process
$$
X_{S,T}(t):=\frac{P(t,S)}{P(t,T)},
$$
with terminal value $X_{S,T}(T)=P(T,S)/P(T,T)$. By taking differentials
under the risk-neutral measure $\mathbb{Q}$ we have
\begin{equation}
\nonumber
\begin{array}{rcl}
\displaystyle d\left( \frac{P(t,S)}{P(t,T)}\right) &=&\displaystyle \frac{1}{P(t,T)}dP(t,S)-\frac{P(t,S)}{(P(t,T))^2}dP(t,T)+dP(t,S)\cdot d\left(\frac{1}{P(t,T)}\right)=\\ 
&=& \{ \dots \} dt + X\left\{ \left(S(t,S)-S(t,T)\right)dW(t)\right\}\\
dX&=& \{ \dots \} dt+X\varsigma(t;T,S)dW(t).
\end{array}
\end{equation}
For the second stage we have used equation (\ref{BondRiskNeutral})
applied to the discount bonds $P(t,S)$ and $P(t,T)$. Recall that
$\mathbb{Q}^T$ is a martingale measure and the multidimensional
Girsanov's Theorem\footnote{See \cite{MR:2004} for a detailed
  discussion of it.}   
which locally induces the change into this $T$-forward measure, does
not affect the difussion coefficient of the initially taken
differential. Therefore we have 
$$
dX_{T,S}(t)=X_{T,S}(t)\varsigma(t;T,S)dW^T(t).
$$
Let us introduce the auxiliary process:
$$
Y_{T,S}(t)=\log X_{T,S}(t)
$$
By means of the multidimensional It\^o Lemma, it is not difficult to
prove that the random variable $Y_{T,S}(T)$ distributes like
$$
Y_{T,S}(T) \sim\mathcal{N}\left( \log \frac{P(0,S)}{P(0,T)}-\frac{1}{2}\vartheta^2(T,S), \vartheta^2(T,S)\right),
$$
where $\vartheta^2(T,S)=\int_0^T \| \varsigma(u;T,S) \|^2 \:du$. Now
the computation of the probability under the $T$-forward measure is
straightforward:
$$
\mathbb{Q}^T(P(T,S)\geq K)=\mathbb{Q}^T(Y_{T,S}(T)\geq \log K)=N(d_-)
$$
For the pending probability $\mathbb{Q}^S$, first, note the following:
$$
\mathbb{Q}^S(P(T,S)\geq K)=\mathbb{Q}^S\left(\frac{P(T,T)}{P(T,S)}\leq
  \frac{1}{K}\right) =\mathbb{Q}^S\left(\log\frac{P(T,T)}{P(T,S)}\leq
  -\log K\right).$$
It is enough to introduce the auxiliary processes,
$$
W_{T,S}(t):=\frac{P(t,T)}{P(t,S)},
$$
and,
$$
Z_{T,S}(t):=\log W_{T,S}(t),
$$
for concluding that $Z_{T,S}(T)$ distributes like
$$
Z_{T,S}(T) \sim\mathcal{N}\left( \log \frac{P(0,T)}{P(0,S)}-\frac{1}{2}\vartheta^2(T,S), \vartheta^2(T,S)\right),
$$
and then
$$
\mathbb{Q}^S(Z_{T,S}(T)\leq -\log K)=N(d_+).
$$
\end{demo}
\begin{corol}
The price at $t=0$ of the put option
$$
h(T)=(K-P(T,S))^+
$$
is given by
\begin{equation}
\label{GaussianForwardPut}
\Pi(h,0)=K P(0,T) N(-d_-)-P(0,S) N(-d_+)
\end{equation}
where the quantities $d_{\pm}$ are completely determined by the identities (\ref{d+-}) to (\ref{GaussianForwardVol2}). 
\end{corol}
\begin{demo}
First consider the difference between the call and the put option at time
$t=0$. Under the risk-neutral martingale measure we know
\begin{equation}
\nonumber
\begin{split}
V-\Pi & =\mathbb{E} \left[ D(0,T) \left\{(P(T,S)-K)^+-(K-P(T,S))^+
  \right\} \right]\\
& =\mathbb{E}\left[ D(0,T)(P(T,S)-K) \right]
\end{split}
\end{equation}
By equation (\ref{ArbitrageFree}) we have $$\mathbb{E}\left[ D(0,T)K
\right]=K P(0,T).$$ However, we have the problem of the correlation
between the discounting factor and the payoff factor for the first
term $$\mathbb{E}\left[ D(0,T)P(T,S) \right].$$ We can circumvent this
problem by using the \emph{Change of Numeraire Theorem} in an
identical way to that shown in Proposition 3. Recall first that the
likelihood
$$
\lambda^S(T)=\frac{d\mathbb{Q}^S}{d\mathbb{Q}}=\frac{P(T,S)/P(0,S)}{B(T)/B(0)}=\frac{D(0,T)P(T,S)}{P(0,S)},
$$
induces the change of the $S$-forward measure into the risk-neutral
measure. Thus we have:
$$
\mathbb{E}\left[ D(0,T)P(T,S) \right]=\mathbb{E}\left[
  \lambda^S(T)P(0,S) \right]=\mathbb{E}^S\left[
  P(0,S) \right],
$$
and then the \emph{Put-Call Parity Relation}:
\begin{equation}
\label{PutCall}
V-\Pi=P(0,S)-KP(0,T),
\end{equation}
is finally inferred.
\end{demo}
% \end{document}% \end{document}% \end{document}% \end{document}% \end{document}









